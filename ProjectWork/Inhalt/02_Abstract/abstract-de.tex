\renewcommand{\abstractname}{Abstract} % Veränderter Name für das Abstract
\begin{abstract}
\begin{addmargin}[1.5cm]{1.5cm}        % Erhöhte Ränder, für Abstract Look
\thispagestyle{plain}                  % Seitenzahl auf der Abstract Seite

\begin{center}
\small\textit{- Deutsch -}             % Angabe der Sprache für das Abstract
\end{center}

\vspace{0.25cm}
In dieser Arbeit werden die Objektdetektoren \textit{You Only Look Once} und \textit{Single Shot MultiBox Detector} nach Präzision, Reaktionsvermögen,  Trainings- und Inferenzverhalten miteinander verglichen und auf deren Potential zum industriellen Einsatz untersucht. Das Hintergrundszenario des \textit{Smart Warehouses} bietet dabei Live-Video Daten einer Drohne mit Warengegenständen in einem Warenhaus, die in Echtzeit klassifiziert und lokalisiert werden sollen. Dadurch sollen in Zukunft in der Industrie Inventuren und Bestandsanalysen eines Warenhauses zeit- und kostengünstig sowie ressourcenschonend ermöglicht werden können.

\vspace{0.25cm}
Diese Machbarkeitsstudie hat zum Ziel herauszufinden, ob das Szenario des \textit{Smart Warehouse} technisch umsetzbar ist. Zusätzlich liegt der Fokus ebenso auf den Objektdetektoren selbst, deren Unterschiede hinsichtlich Architektur, Verhalten und wie gut sie allgemein für industrielle Anwendungsszenarien grundsätzlich geeignet sind. 

\end{addmargin}
\end{abstract}