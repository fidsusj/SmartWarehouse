\chapter*{Formelverzeichnis}
\addcontentsline{toc}{chapter}{Formelverzeichnis} % Hinzufügen zum Inhaltsverzeichnis 

% Definition des neuen Befehls für das Einfügen der Abkürzung der Einheit
\newcommand{\acrounit}[1]{
  \acroextra{\makebox[18mm][l]{\si[per=frac,fraction=nice]{#1}}}
}
\begin{acronym}[dmin] % längstes Kürzel wird verw. für den Abstand zw. Kürzel u. Text

	%Lateinische
	%	Bezeichnungen erscheinen vor den griechischen Abkürzungen, die keine Formelzeichen darstellen, sind
	%	getrennt aufzuführen und außerdem beim ersten Gebrauch einzuführen.

	%Wichtige Formeln werden jeweils bei ihrem ersten Auftreten durch eingeklammerte Zahlen am Ende oder
	%Anfang der Zeile gekennzeichnet. Die gewählte Kennzeichnung ist in der gesamten Arbeit einheitlich beizubehalten.

	% Alphabetisch selbst sortieren - nicht verwendete Formeln rausnehmen!
	% Allgemein: \acro{KÜRZEL}[ABKÜRZUNG]{\acrounit{SI-EINHEIT}BESCHREIBUNG}

\end{acronym}