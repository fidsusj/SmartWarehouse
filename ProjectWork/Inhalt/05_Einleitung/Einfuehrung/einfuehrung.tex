\chapter{Einführung}

\section{Forschungsumfeld}

Einen Teilbereich des maschinellen Lernens (engl.: machine learning) stellt das \textit{Deep Learning} dar, welches auf künstlichen neuronalen Netzen (engl.: artificial neural networks) (ANNs) basiert \cite[S. 253]{AurelienGeron.2018}. Unter einer Vielzahl von Typen von ANNs wie Autoencodern, Deep Boltzmann Machines oder rekurrenten neuronale Netzen befindet sich ebenso die Klasse der \textit{Convolutional Neural Networks} (CNNs), welche hauptsächlich zur Lösung von Klassifikationsproblemen in der Audio-, Text- und Bildverarbeitung genutzt werden \cite{MathWorks.2019}.

Ein Forschungsfeld im \textit{Deep Learning} stellen Objektdetektoren dar, welche basierend auf CNNs neben Bildklassifikationsproblemen ebenso in der Lage sind Lokalisationsprobleme zu lösen. Solchen Objektdetektoren werden in der heutigen Zeit immer mehr Bedeutung zugesprochen angesichts neuer Herausforderungen wie autonomen Fahren, automatisierter industrieller Verarbeitung oder aber auch staatlicher Überwachung. Verschiedene Ansätze werden zur Realisierung von Objektdetektoren verwendet, unter anderem Netzarchitekturen wie \textit{You Only Look Once} (YOLO), \textit{Single Shot MultiBox Detector} (SSD) oder \textit{Regional Convolutional Neural Networks} (R-CNN). 

Wie sich die unterschiedlichen Objektdetektoren unter Echtzeitvoraussetzungen im Betrieb verhalten, soll anhand des Industriebeispiels \textit{Smart Warehouse} innerhalb dieser Arbeit untersucht werden. 
