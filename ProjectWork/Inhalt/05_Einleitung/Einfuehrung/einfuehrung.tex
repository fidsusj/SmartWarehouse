\chapter{Einführung}

\section{Forschungsumfeld}

Einen Teilbereich des maschinellen Lernens (engl.: machine learning) stellt das \textit{Deep Learning} dar, welches auf künstlichen neuronalen Netzen (engl.: artificial neural networks) (ANNs) basiert \cite{AurelienGeron.2018}. Unter einer Vielzahl von Typen von ANNs wie Autoencodern, Deep Boltzmann Machines oder rekurrenten neuronale Netzen befindet sich ebenso die Klasse der \textit{Convolutional Neural Networks} (CNNs), welche hauptsächlich zur Lösung von Klassifikationsproblemen in der Audio-, Text- und Bildverarbeitung genutzt werden \cite{MathWorks.2019}.

Ein Forschungsfeld im \textit{Deep Learning} stellen Objektdetektoren dar, welche basierend auf CNNs neben Bildklassifikationsproblemen ebenso in der Lage sind, Lokalisationsprobleme zu lösen. Solchen Objektdetektoren werden in der heutigen Zeit immer mehr Bedeutung zugesprochen angesichts neuer Herausforderungen wie autonomen Fahren, automatisierter industrieller Verarbeitung oder aber auch staatlicher Überwachung. Verschiedene Ansätze werden zur Realisierung von Objektdetektoren verwendet, unter anderem Netzarchitekturen wie \textit{You Only Look Once} (YOLO) oder \textit{Single Shot MultiBox Detector} (SSD). 

Gerade in Zeiten des industriellen Wandels in Richtung \textit{Industrie 4.0} können solche Objektdetektoren ein großes Optimierungspotential für bestehende Industrieszenarien bieten, beispielsweise in der Lagerhaltung und Logistik. Kombiniert mit einer autonomen Drohne können Objektdetektoren es ermöglichen, ohne menschliche Hilfe Inventuren und Bestandsprüfungen in einem Lager- oder Warenhaus durchzuführen. Start-up Unternehmen wie \textit{doks. innovation} werben bereits mit ähnlichen Lösungen, die 80\% Zeiteinsparung und 90\%Kostensenkung versprechen \cite{doks.innovation.2019}. Lösungen wie \textit{inventAIRyX} beschränken sich allerdings speziell auf Lagerhäuser, in denen die verpackten Waren mittels Sensoren identifiziert werden, was Großhändler mit Warenhäusern wie \textit{Baumarkt} oder \textit{Selgros} ausschließt. Statt Waren mittels RFID Chips oder Barcodes zu identifizieren, soll in dieser Arbeit der Einsatz von Objektdetektoren für dieses Szenario evaluiert werden.

Wie sich die unterschiedlichen Objektdetektoren unter Echtzeitvoraussetzungen im Betrieb verhalten, soll anhand des Industriebeispiels \textit{Smart Warehouse} innerhalb dieser Arbeit untersucht werden. 

