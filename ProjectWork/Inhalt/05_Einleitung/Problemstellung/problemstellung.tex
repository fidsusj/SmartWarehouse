\section{Problemstellung und Motivation}

Das \textit{Smart Warehouse} beschreibt ein Warenhaus, welches unter Einsatz einer Drohne in der Lage sei soll, Inventuren und Bestandsprüfungen weitgehend ohne menschliche Hilfe durchzuführen. Das Live-Bild der Drohne soll von den Objektdetektoren dazu genutzt werden, Warengegenstände zu lokalisieren und zu klassifizieren. 

Neben der Frage, ob ein solches Industrieszenario überhaupt umsetzbar ist, sollen die Objektdetektoren in diesem Anwendungsszenario nach verschiedenen Kriterien miteinander verglichen und beurteilt werden. Diese Kriterien lassen sich hauptsächlich in die Kategorien Präzision, Reaktionsvermögen, Trainings- und Inferenzverhalten untergliedern und werden genauer eingeführt. Dadurch lassen sich Aussagen darüber treffen, ob nach dem momentanen Forschungsstand um Objektdetektoren solche das Potential bieten, industriell eingesetzt zu werden. 

Falls die Machbarkeitsstudie des \textit{Smart Warehouse} glückt, so kann der Industrie ein kostengünstiges, zeitsparendes und ressourcenschonendes Modell zur Inventurverwaltung eines Warenhauses angeboten werden.