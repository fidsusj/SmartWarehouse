\section{Vorgehensweise und Zielsetzung}

Im Grundlagenkapitel II muss sich mit den theoretischen Grundlagen von CNNs und Objektdetektoren auseinander gesetzt werden. Hierzu ist zunächst eine Einführung in neuronale Netz erforderlich, darunter zu Perzeptronen, dem Gradientenverfahren, dem Backpropagation Algorithmus und Hyperparametern zum Trainieren eines neuronalen Netzes (siehe Kapitel 2.1 und 2.2). 

Nachdem kurz auf den Grundbaustein moderner Objektdetektoren eingegangen wird, den CNNs, können anschließend die Funktionsweisen und Architekturen der zwei miteinander verglichenen Objektdetektoren \textit{YOLO} und \textit{SSD} erläutert werden (siehe Kapitel 2.3). Bei \textit{YOLO} ist zu bemerken, dass unterschiedliche Evolutionsstufen der drei Detektoren zu betrachten sind.

Um weitere Grundlagen zum Umfeld während des Trainierens von neuronalen Netzen einzuführen, wird anschließend über die Anforderungen eines Datensatzes gesprochen (siehe Kapitel 2.4), bevor Trainingsinfrastrukturen in der Cloud vorgestellt werden (siehe Kapitel 2.5). 

In der Konzeptionsphase in Kapitel III sollen zunächst die Vergleichskriterien eingeführt werden und deren Metriken anschließend für initiale Benchmarkdatensätze für jeden Objektdetektor ermittelt werden. Hierzu wird auf die Datensätze \textit{Pascal Visual Object Classes} (PascalVOC), \textit{Common Objects in Context} (COCO) und ImageNet zurückgegriffen. Anschließend wird der Datensatz für das \textit{Smart Warehouse} Szenario eingeführt. 

In der Realisierung werden die Herausforderungen zur Steuerung und Anbindung der Drohne betrachtet und zudem die Objektdetektoren auf die realen Datensätze trainiert. Auch die Entwicklung der Webapplikation zur Visualisierung des Live-Bildes und der erkannten Objekte wird Bestandteil dieses Kapitels sein. Die Ergebnisse der Realisierungsphase werden im folgenden Kapitel dargestellt. 

Ziel der Arbeit ist es Aussagen über die Fähigkeit von Objektdetektoren zum Einsatz in der Industrie zu treffen, indem eine Bewertung der Verhaltensweisen der Objektdetektoren nach den eingeführten Bewertungskriterien durchgeführt wird. Auch wirtschaftliche Gesichtspunkte werden in diesem Kapitel nicht außer Acht gelassen. 

Zuletzt wird das Wesen der Arbeit nochmals kurz zusammengefasst und anschließend auf mögliche Verbesserungen und Ausblicke in die Zukunft aufmerksam gemacht. 
