\section{Vorgehensweise und Zielsetzung}

Im Grundlagenkapitel 2 muss sich mit den theoretischen Grundlagen von CNNs und Objektdetektoren auseinander gesetzt werden. Hierzu ist zunächst eine Einführung in \textit{Deep Learning} zur Bildverarbeitung und neuronale Netz erforderlich als auch zu Hyperparametern zum Trainieren eines neuronalen Netzes (siehe Kapitel \ref{bildverarbeitung}, \ref{anns} und \ref{hyperparam}).

Um weitere Grundlagen zum Training von neuronalen Netzen einzuführen, wird anschließend über die Anforderungen eines Datensatzes gesprochen (siehe Kapitel \ref{data}), bevor weitere Grundlagen zu Objektdetektoren eingeführt werden (siehe Kapitel \ref{basics}). 

Nachdem zu Beginn des Kapitels \ref{basics} kurz auf den Grundbaustein moderner Objektdetektoren eingegangen wird, den CNNs, können anschließend die Funktionsweisen und Architekturen der drei miteinander verglichenen Objektdetektoren der \textit{R-CNN} Familie, \textit{YOLO} und des \textit{SSDs} erläutert werden (siehe Kapitel \ref{detection}). Bei \textit{R-CNN} und \textit{YOLO} ist zu bemerken, dass unterschiedliche Evolutionsstufen der Detektoren zu betrachten sind.

Um weitere Grundlagen zum Training von neuronalen Netzen einzuführen, wird anschließend über zwei wesentlichen Speicherformate eines Datensatzes gesprochen (siehe Kapitel \ref{format}), bevor verschiedene Cloud Anbieter für das Trainieren von \textit{Deep Learning} Modellen aufgezeigt werden (siehe Kapitel \ref{cloud}). 

Zuletzt wird ein kurzer Überblick über ausgewählte Drohnenangebote gegeben und dabei auch ein Einblick in die gesetzlichen Rahmenbedingungen zu Drohnen gegeben (siehe Kapitel \ref{drohnen})

In Kapitel 3 werden chronologisch Teilziele der Konzeption beschrieben, darunter das Erstellen eines Trainingsdatensatzes (siehe Kapitel \ref{traindata}), dem Einführen von Bewertungskriterien (siehe Kapitel \ref{eval}), der Auswahl von Objektdetektoren, der Trainingsinfrastruktur und der Drohne (siehe Kapitel \ref{detect}, \ref{infrastructure}, \ref{drone_selection}) und die Spezifikation der Inventursoftware (siehe Kapitel \ref{software}).

In der Realisierung werden die Herausforderungen zur Steuerung und Anbindung der Drohne betrachtet und zudem die Objektdetektoren auf die realen Datensätze trainiert. Auch die Entwicklung der Webapplikation zur Visualisierung des Live-Bildes und der erkannten Objekte sowie der darin realisierte Zählalgorithmus zur Durchführung der Inventur wird Bestandteil dieses Kapitels sein. Die Ergebnisse der Realisierungsphase werden nach den in Unterkapitel \ref{eval} definierten Bewertungskriterien im folgenden Kapitel dargestellt. 

Ziel der Arbeit ist es Aussagen über die Fähigkeit von Objektdetektoren zum Einsatz in der Industrie zu treffen, indem eine Bewertung der Verhaltensweisen der Objektdetektoren nach den eingeführten Bewertungskriterien durchgeführt wird. Als Umgebung und Rahmenszenario für die Evaluation dient das \textit{Smart Warehouse} Szenario, dessen Machbarkeit ebenfalls herausgestellt werden soll (siehe Kapitel \ref{evaluation}).

Zuletzt wird das Wesen der Arbeit nochmals kurz zusammengefasst und anschließend auf mögliche Verbesserungen und Ausblicke in die Zukunft aufmerksam gemacht. 
