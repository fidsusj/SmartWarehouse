\section{Vorgehensweise und Zielsetzung}

Im Grundlagenkapitel II muss sich mit den theoretischen Grundlagen von CNNs und Objektdetektoren auseinander gesetzt werden. Hierzu ist zunächst eine Einführung in Deep Learning zur Bildverarbeitung und neuronale Netz erforderlich, darunter zu Perzeptronen, dem Gradientenverfahren, dem Backpropagation Algorithmus und Hyperparametern zum Trainieren eines neuronalen Netzes (siehe Kapitel 2.1, 2.2 und 2.3). 

Um weitere Grundlagen zum Training von neuronalen Netzen einzuführen, wird anschließend über die Anforderungen eines Datensatzes gesprochen (siehe Kapitel 2.4), bevor die eigentlichen Objektdetektoren eingeführt werden (siehe Kapitel 2.5). 

Nachdem kurz auf den Grundbaustein moderner Objektdetektoren eingegangen wird, den CNNs, können anschließend die Funktionsweisen und Architekturen der drei miteinander verglichenen Objektdetektoren der \textit{R-CNN} Familie, \textit{YOLO} und des \textit{SSDs} erläutert werden. Bei \textit{R-CNN} und \textit{YOLO} ist zu bemerken, dass unterschiedliche Evolutionsstufen der Detektoren zu betrachten sind.

Schließlich werden verschiedene Cloud Anbieter für das Trainieren von \textit{Deep Learning} Modellen aufgezeigt (siehe Kapitel 2.6).

In Kapitel III werden chronologisch Teilziele der Konzeption beschrieben, darunter das Erstellen eines Trainingsdatensatzes (siehe Kapitel 3.1), dem Einführen von Bewertungskriterien (siehe Kapitel 3.2), der Auswahl von Objektdetektoren, der Trainingsinfrastruktur und der Drohne (siehe Kapitel 3.3, 3.4, 3.4) und die Spezifikation der Inventursoftware (siehe Kapitel 3.6).

In der Realisierung werden die Herausforderungen zur Steuerung und Anbindung der Drohne betrachtet und zudem die Objektdetektoren auf die realen Datensätze trainiert. Auch die Entwicklung der Webapplikation zur Visualisierung des Live-Bildes und der erkannten Objekte wird Bestandteil dieses Kapitels sein. Die Ergebnisse der Realisierungsphase werden im folgenden Kapitel dargestellt. 

Ziel der Arbeit ist es Aussagen über die Fähigkeit von Objektdetektoren zum Einsatz in der Industrie zu treffen, indem eine Bewertung der Verhaltensweisen der Objektdetektoren nach den eingeführten Bewertungskriterien durchgeführt wird. Dies soll mit Hilfe der Umsetzung des \textit{SmartWarehouse} Szenarios bewiesen werden.

Zuletzt wird das Wesen der Arbeit nochmals kurz zusammengefasst und anschließend auf mögliche Verbesserungen und Ausblicke in die Zukunft aufmerksam gemacht. 
