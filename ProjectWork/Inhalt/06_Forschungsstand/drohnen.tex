\section{Drohnen} \label{drohnen}

Was die gesetzlichen Anforderungen zu Drohnen betrifft, so wurden im Juni 2019 von der \textit{European Aviation Safety Agency} (EASA) einheitliche Regeln veröffentlicht, die den Drohnenbetrieb in der Europäischen Union einheitlich regeln sollen. Da den Mitgliedsstaaten ein Zeitraum von einem Jahr zur Umsetzung der Regularien zugesprochen wurde, gelten in Deutschland weiterhin die Vorschriften der deutschen Drohnen-Verordnung von 2017 \cite{EASA.2019}.

Diese schreiben unter anderem vor \cite{Drohnen.de.2020}:
\begin{itemize}
	\item Eine Kennzeichnungspflicht ab einem Startgewicht von über 250g,
	\item Eine maximale Flughöhe von 100 Metern über dem Grund,
	\item Eine Haftpflichtversicherung und
	\item Flugverbotszonen (Wohngrundstücke, Naturschutzgebiete, etc.).
\end{itemize}

Analysiert man den Markt auf programmierbare Drohnen mit einem frei zugänglichen \textit{Software Development Kit} (SDK) und integrierter Kamera, so fällt das Angebot sehr gering aus. Im Folgenden soll ein Überblick über die zwei verfügbaren Modelle \textit{Ryze Tech Tello EDU} und \textit{Parrot Bebop 2} gegeben werden \cite{RyzeRobotics.2020, Parrot.20200520}.

\begin{center}
	\begin{tabular}[H]{l|c|c}
		& Ryze Tech Tello EDU & Parrot Bebop 2\\
		\hline
		Gewicht in Gramm & 87 & 500 \\
		Maximale Bildgröße & 2592x1936 & 4096x3072 \\
		Video Aufnahme Modi & 1280x720 30p & 1920×1080 30p \\
		Batterie Kapazität in mAh & 1100 & 2700 \\
		Max. Flugzeit in Minuten & 13 & 23 \\
		Max. Fluggeschwindigkeit in km/h & 28.8 & 60 \\
		Flugstabilisierung & Ja & Nein \\
		Preis in € & 159 & 295
	\end{tabular}
	\captionof{table}{Technische Daten zu programmierbaren Drohnen mit Kameraintegration}
	\label{table:drohnenspecs}
\end{center}

Der technische Vergleich (siehe Tabelle \ref{table:drohnenspecs}) bildet die Grundlage für die spätere Auswahl einer Drohne für die Umsetzung des \textit{SmartWarehouse} Szenarios.