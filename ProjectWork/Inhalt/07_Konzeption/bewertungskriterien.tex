\section{Einführen von Bewertungskriterien}

Um Objektdetektoren miteinander vergleichbar zu machen und um deren Potential zum industriellen Einsatz zu bewerten, müssen konkrete Bewertungskriterien eingeführt werden.

\subsection*{Präzision}

Zur Messung der Präzision wird die Metrik \textit{mAP} verwendet. Dies garantiert eine gute Vergleichbarkeit mit den veröffentlichten Leistungsmerkmalen der Objektdetektoren.

\subsection*{Reaktionsvermögen}

Um eine Verarbeitung in Echtzeit zu ermöglichen, muss gewährleistet sein, dass die Inferenzgeschwindigkeit mit dem Modell mit der eingehenden Bildrate einhergeht. Als Maßstab dafür dient die \textit{Frames Per Second} (FPS) Zahl. Echtzeitfähigkeit in der Machbarkeitsstudie ist so definiert, dass die Inferenz mit dem Modell mindestens so schnell ablaufen muss, dass Änderungen in der Umgebung rechtzeitig von Objektdetektor noch wahrgenommen werden können. 

\subsection*{Trainingsverhalten}

Unter dem Punkt Trainingsverhalten wird zusammengefasst, wie schnell sich die einzelnen Modelle mit den unterschiedlichen Objektdetektoren trainieren lassen. Hierbei wird besonderer Fokus darauf gelegt, wie viele Trainingsepochen notwendig sind, bis der Gradient der Fehlerfunktion des neuronalen Netzes keine merkenswerten Fortschritte auf Basis des verwendeten Datensatzes mehr erzielt. Es soll aber auch betrachtet werden, wie mit doppelt erkannten Objekten während des Trainingsprozesses umgegangen wird. 

\subsection*{Inferenzverhalten}

Im Zuge der Evaluation des Inferenzverhaltens werden drei Kriterien betrachtet. 

\begin{itemize}
	\item Das Verhalten bei besonderen Beleuchtungsverhältnissen wie unterbeleuchteten oder überbeleuchteten Gegenden.
	\item Das Verhalten bei extremen Blicklagen auf Basis der Entfernung und des Winkels zum detektierenden Objekt.
	\item Das Verhalten bei nicht vollständig sichtbaren Objekten, z.B. bei Verdeckung.
\end{itemize}
