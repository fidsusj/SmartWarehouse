\chapter{Bewertung}

\section{Einsatz der Objektdetektoren in der Industrie}

Die Bewertung der beiden Detektoren für den industriellen Einsatz baut auf den in Kapitel 3.2 eingeführten Bewertungskriterien auf. 

Hinsichtlich der Präzision fallen beide Objektdetektoren besser als die ursprünglichen Referenzergebnisse aus den wissenschaftlichen Veröffentlichungen aus. Es ist aber zu bemerken, dass die Ergebnisse nicht gut miteinander vergleichbar sind, da sie auf unterschiedlichen Daten trainiert wurden. Der Vergleichsdatensatz ist hierbei \textit{PascalVOC 2007}, der 20 Klassen besitzt und damit für das Modell komplexer zu modellieren ist, als der \textit{SmartWarehouse} Datensatz mit nur 9 Klassen. Mit 83,1\% fällt \textit{SSD300} um 8,8\% besser aus, während \textit{YOLOv3} sogar eine Verbesserung von 16.7\% verzeichnet. Dennoch lässt sich darauf schließen, dass beide Implementierungen bezüglich Präzision in einer Größenordnung skalieren, die dem Niveau der Vergleichsergebnisse aus \textit{PascalVOC 2007} nachkommt. Die leicht erhöhte Präzision von \textit{SSD300} gegenüber \textit{YOLOv3} liegt daran, dass \textit{SSD300} im Gegensatz zu \textit{YOLOv3} Bounding Box Vorschläge unterschiedlicher Seitenverhältnisse zulässt und die Unterteilung in Gitterstrukturen ebenfalls für mehrere Skalierungen durchgeführt wird.

Ein weiteres Bewertungskriterium ist das Reaktionsvermögen. Hier schneidet \textit{SSD} mit 30 FPS deutlich besser ab als \textit{YOLO} mit ?? FPS. [TBD]

TODO:
- Reaktionsvermögen YOLO

Bezüglich Trainingsverhalten könnte \textit{SSD300} das Training mit leistungsfähigeren Grafikkarten wie der \textit{NVIDIA Tesla V100} von 16,5 Stunden auf umgerechnet 5,2 Stunden rund drei Mal schneller vollziehen. Die Rechnung legt die Tabelle \ref{table:comparison} zugrunde. Pro Sekunde kann eine \textit{NVIDIA GTX 1080} hierbei $2560 \cdot 8,873 = 22.714,88 teroFLOPs$ vollziehen im Vergleich zu $5120 \cdot 14,13 = 72.345,6 TeraFLOPs$ pro Sekunde bei einer \textit{NVIDIA Tesla V100}. Da die \textit{NVIDIA Tesla V100} zusätzlich noch über Tensor Cores verfügt, kann der Zeitgewinn sogar noch höher eingeschätzt werden. Bei acht \textit{NVIDIA Tesla V100} wie in Abbildung \ref{tpu} betragen die Trainingskosten nur noch ca. 39 Minuten. Auch hier ist zu bemerken, dass das Ergebnis nicht mit den in Abbildung \ref{tpu} angeführten 216 Minuten zu vergleichen ist, da nicht nur die Netzstruktur bei \textit{SSD300} einfacher ist als bei \textit{ResNet-50} mit 50 \textit{hidden Layern}, sondern vermutlich auch der Benchmark Datensatz größer war\footnote{Hier fehlen in der referenzierten Quelle die Angaben zum Referenzdatensatz}. Dennoch lässt sich festhalten, dass sich \textit{SSD300} durchaus mit aktuellen Cloud Grafikkarten effizient trainieren lassen lässt. Da die Custom-Implementierung allerdings Hilfsstrukturen auf dem Sekundärspeicher erstellt, können hierbei beim Trainieren auf der Cloud Probleme auftreten. Meistens wird der Sekundärspeicher auf virtuellen Maschinen in der Cloud zum Schutz vor Angriffen nur \textit{read-only} bereitgestellt. 

TODO:
- Bewertung Trainingsverhalten YOLO

Zuletzt muss das Inferenzverhalten betrachtet werden. 

- beleuchtung
- blicklagen und entfernung
- verdeckung
- doppelt erkannt




\section{Machbarkeit des SmartWarehouse Szenarios}

Auswahl YOLO SSD

\begin{itemize}
	\item Eignet sich YOLO und SSD für die Industrie?
	\item Ist das SmartWarehouse Szenario umsetzbar?
\end{itemize}

