\chapter{Zusammenfassung und Ausblick}

Ziel der Arbeit war es zum einen zu evaluieren, wie gut die bestehenden Objektdetektoren für industrielle Anwendungsszenarien grundsätzlich geeignet sind, zum anderen, ob das spezifische Anwendungsszenario zur Durchführung einer Inventur von Warenhäusern mit einer Drohne prototypisch umsetzbar ist. Nach dieser Arbeit sind nun folgende Erkenntnisse festzuhalten.

Zunächst ist evaluiert worden, dass Objektdetektoren wie \textit{SSD} und \textit{YOLO} durchaus das Potential ausweisen, industriell genutzt zu werden. Nicht nur hinsichtlich ihrer Präzision im Detektionsvermögen weisen sie zuversichtliche Ergebnisse auf, sondern auch in der Schnelligkeit ihrer Detektion. Beide weisen bei speziellen Detektionsszenarien ihre Stärken und Schwächen auf. Gerade die Detektion von teilweise verdeckten Objekten stellt nach wie vor ein Problem dar, für das alternative Lösungswege gefunden werden müssen. 

Daneben stellt sich die Machbarkeitsstudie \textit{SmartWarehouse} unter bestimmten Umständen als machbar heraus. Es wurde eine Drohne programmiert, die ein Miniaturmodell eines Warenhauses durchfliegt. Ihr Drohnenstream wurde an einen Server zur Inferenz mit einem Objektdetektionsmodell mit \textit{YOLO} gesendet und die daraus resultierenden Objekte gezählt. Das Ergebnis und der Live-Stream der Inferenz wurden auf einer Webapplikation dargestellt. Offen bleibt das Problem einer doppelten Detektion von Objekten bei mehrfachen Erscheinen im Bild nach gewissen zeitlichen Abständen, da keine eindeutige Identifizierung von Objekten mit reiner Objektdetektion möglich ist. Damit wäre das Zählen von gleichen Objekten an unterschiedlichen Lagerplätzen mit Hilfe eines Objektdetektors nicht umsetzbar, zumindest nicht ohne weitere Zuhilfenahme von zum Beispiel einem GPS Sensor.

Um das \textit{SmartWarehouse} Szenario weiter auszubauen, soll als nächstes der Datensatz vom simplen Beispiel von Getränkeflaschen auf eine echte Warenhausumgebung erweitert werden. Dadurch wird ebenfalls erhofft, die Präzision des Modells weiter zu verbessern und Inkonsistenzen bei unterschiedlichen Beleuchtungsverhältnissen, extremen Blicklagen, größeren Entfernungen und unterschiedlichen Verdeckungsgraden zu beheben. Zudem muss die Flugsequenz der Drohne dahingehend optimiert werden, bei Stellen starker Verdeckungsgrade alternative Betrachtungswinkel zu wählen. 

Um spezielle Rahmenbedingungen bei der Zählung durch Objektdetektion auszuschließen, könnte das Szenario ebenso zur eindeutigen Identifizierung von Objekten mit konventionellen RFID Chips erweitert werden. Dies ermöglicht ebenso neue Szenarien zu entwickeln, in denen Objektdetektion nur Suche von Objekten einer bestimmten Klasse genutzt wird und deren Identität anschließend mit RFID Chips sichergestellt wird. 