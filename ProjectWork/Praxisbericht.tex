% ------------------------------------------------------------
% LaTeX Template für die DHBW zum Schnellstart!
% Original: https://github.wdf.sap.corp/D064996/LaTeX-Template-DHBW
% ------------------------------------------------------------
% ---- Präambel mit Angaben zum Dokument
\input{Inhalt/00_Latex/praeambel}

% ---- Elektronische Version oder Gedruckte Version?
% ---- Unterschied: Die elektronische Version enthält keinen Platzhalter für die Unterschrift
\usepackage{ifthen}
\newboolean{e-Abgabe}
\setboolean{e-Abgabe}{true}    % false=gedruckte Fassung

% ---- Persönlichen Daten:
\newcommand{\titel}{Machbarkeitsstudie: Smart Warehouse}
\newcommand{\subtitel}{Echtzeit-Objektdetektoren im Vergleich}
\newcommand{\titelheader}{Smart Warehouse}
\newcommand{\arbeit}{Studienarbeit}
\newcommand{\studiengang}{Angewandte Informatik}
\newcommand{\studienjahr}{2019}
\newcommand{\autor}{Felix Hausberger und Robin Kuck}
\newcommand{\autorReverse}{Hausberger, Felix und Kuck, Robin}
\newcommand{\verfassungsort}{Karlsruhe}
\newcommand{\matrikelnr}{2773463, 4409176}
\newcommand{\kurs}{TINF17B2}
\newcommand{\bearbeitungsmonat}{Oktober 2019 - Mai 2020}
\newcommand{\abgabe}{18. Mai 2020}
\newcommand{\bearbeitungszeitraum}{30.09.2019 - 18.05.2020}
\newcommand{\firmaName}{SAP SE}
\newcommand{\firmaStrasse}{Dietmar-Hopp-Allee 16}
\newcommand{\firmaPlz}{69190 Walldorf, Deutschland}
\newcommand{\betreuerFirma}{}
\newcommand{\betreuerDhbw}{PD Dr.-Ing. Markus Reischl}

\input{Inhalt/00_Latex/kopfundFusszeile}

% ---- Hilfreiches
\newcommand{\zB}{z.\,B. }   % "z.B." mit kleinem Leeraum dazwischen (ohne wäre nicht korrekt)
\newcommand{\dash}{d.\,h. }

% ---- Silbentrennung (falls LaTeX defaults falsch / nicht gewünscht sind)
\hyphenation{HANA}         % anstatt HA-NA
\hyphenation{Graph-Script} % anstatt GraphS-cript

% ---- Beginn des Dokuments
\begin{document}
\setlength{\parindent}{0pt}              % Keine Paragraphen Einrückung.
                                         % Dafür haben wir den Abstand zwischen den Paragraphen.
\setcounter{secnumdepth}{2}              % Nummerierungstiefe fürs Inhaltsverzeichnis
\setcounter{tocdepth}{1}                 % Tiefe des Inhaltsverzeichnisses. Ggf. so anpassen,
                                         % dass das Verzeichnis auf eine Seite passt.
\sffamily                                % Serifenlose Schrift verwenden.

% ---- Vorspann
% ------ Titelseite
\singlespacing
\include{Inhalt/01_Standard/titelseite}  % Titelseite
\newcounter{savepage}
\pagenumbering{Roman}                    % Römische Seitenzahlen
\onehalfspacing

% ------ Erklärung, Sperrvermerk, Abstact
\chapter*{Eidesstattliche Erklärung}
Wir versicheren hiermit, dass wir unsere \arbeit{} mit dem Thema:
\begin{quote}
	\textit{\titel: Smart Warehouse}
\end{quote} 
gemäß § 5 der \enquote{Studien- und Prüfungsordnung DHBW Technik} vom 29. September 2017 selbstständig verfasst und keine anderen als die angegebenen Quellen und Hilfsmittel benutzt haben. Die Arbeit wurde bisher keiner anderen Prüfungsbehörde vorgelegt und auch nicht veröffentlicht.

\vspace{0.25cm}

Wir versicheren zudem, dass die eingereichte elektronische Fassung mit der gedruckten Fassung übereinstimmt.

\vspace{1cm}

\verfassungsort, den 24. Mai 2020 \\[0.5cm]
\ifthenelse{\boolean{e-Abgabe}}
	{\underline{Gez. \autor}}
	{\makebox[6cm]{\hrulefill}}\\ 
\autorReverse

\renewcommand{\abstractname}{Abstract} % Veränderter Name für das Abstract
\begin{abstract}
\begin{addmargin}[1.5cm]{1.5cm}        % Erhöhte Ränder, für Abstract Look
\thispagestyle{plain}                  % Seitenzahl auf der Abstract Seite

\begin{center}
\small\textit{- English -}             % Angabe der Sprache für das Abstract
\end{center}

\vspace{0.25cm}
In this thesis the object detectors \textit{You Only Look Once} and \textit{Single Shot MultiBox Detector} are compared for precision, reactivity, training and inference behaviour and examined for their potential for industrial use. The background scenario of the \textit{Smart Warehouse} offers live video data of a drone with goods in a warehouse, which are to be classified and localized in real time. In the future, this should make it possible to carry out inventories and inventory analyses of a warehouse in a time- and cost-efficient manner conserving resources.

\vspace{0.25cm}
The goal of this feasibility study is to find out whether the \textit{Smart Warehouse} scenario is technically feasible. In addition, the focus is also on the object detectors themselves, their differences in architecture, behavior and how well they are generally suitable for industrial application scenarios.

\end{addmargin}
\end{abstract}
\renewcommand{\abstractname}{Abstract} % Veränderter Name für das Abstract
\begin{abstract}
\begin{addmargin}[1.5cm]{1.5cm}        % Erhöhte Ränder, für Abstract Look
\thispagestyle{plain}                  % Seitenzahl auf der Abstract Seite

\begin{center}
\small\textit{- Deutsch -}             % Angabe der Sprache für das Abstract
\end{center}

\vspace{0.25cm}
In dieser Arbeit werden die Objektdetektoren \textit{You Only Look Once} und \textit{Single Shot MultiBox Detector} nach Präzision, Reaktionsvermögen,  Trainings- und Inferenzverhalten miteinander verglichen und auf deren Potential zum industriellen Einsatz untersucht. Das Hintergrundszenario des \textit{Smart Warehouses} bietet dabei Live-Video Daten einer Drohne mit Warengegenständen in einem Warenhaus, die in Echtzeit klassifiziert und lokalisiert werden sollen. Dadurch sollen in Zukunft in der Industrie Inventuren und Bestandsanalysen eines Warenhauses zeit- und kostengünstig sowie ressourcenschonend ermöglicht werden können.

\vspace{0.25cm}
Diese Machbarkeitsstudie hat zum Ziel herauszufinden, ob das Szenario des \textit{Smart Warehouse} technisch umsetzbar sowie wirtschaftlich sinnvoll ist. Zusätzlich liegt der Fokus ebenso auf den Objektdetektoren selbst und deren Unterschiede hinsichtlich Architektur und Verhalten im \textit{Smart Warehouse} Umfeld. 

\end{addmargin}
\end{abstract}

% ------ Inhaltsverzeichnis
\singlespacing
\tableofcontents

% ------ Verzeichnisse
\renewcommand*{\chapterpagestyle}{plain}
\pagestyle{plain}
\setlength{\cftsecnumwidth}{3em}
\setlength{\cfttabnumwidth}{3em}            
%\chapter*{Formelverzeichnis}
\addcontentsline{toc}{chapter}{Formelverzeichnis} % Hinzufügen zum Inhaltsverzeichnis 

% Definition des neuen Befehls für das Einfügen der Abkürzung der Einheit
\newcommand{\acrounit}[1]{
  \acroextra{\makebox[18mm][l]{\si[per=frac,fraction=nice]{#1}}}
}
\begin{acronym}[dmin] % längstes Kürzel wird verw. für den Abstand zw. Kürzel u. Text

	%Lateinische
	%	Bezeichnungen erscheinen vor den griechischen Abkürzungen, die keine Formelzeichen darstellen, sind
	%	getrennt aufzuführen und außerdem beim ersten Gebrauch einzuführen.

	%Wichtige Formeln werden jeweils bei ihrem ersten Auftreten durch eingeklammerte Zahlen am Ende oder
	%Anfang der Zeile gekennzeichnet. Die gewählte Kennzeichnung ist in der gesamten Arbeit einheitlich beizubehalten.

	% Alphabetisch selbst sortieren - nicht verwendete Formeln rausnehmen!
	% Allgemein: \acro{KÜRZEL}[ABKÜRZUNG]{\acrounit{SI-EINHEIT}BESCHREIBUNG}

\end{acronym}
\chapter*{Abkürzungsverzeichnis}
\addcontentsline{toc}{chapter}{Abkürzungsverzeichnis} % Hinzufügen zum Inhaltsverzeichnis 

\begin{acronym}[WYSISWG] % längstes Kürzel wird verw. für den Abstand zw. Kürzel u. Text

	% Alphabetisch selbst sortieren - nicht verwendete Kürzel rausnehmen!
	% Keine Umgangssprachlichen Abkürzungen
	%Lateinisch vor griechisch
	
	\acro{ANN}{Artificial Neural Network}
	\acro{CNN}{Convolutional Neural Network}
	\acro{COCO}{Common Objects in Context}
	\acro{IoU}{Intersection over Union}
	\acro{LReLU}{Leaky Rectified Linear Unit}
	\acro{PReLU}{Parametric Rectified Linear Unit}
	\acro{PascalVOC}{Pascal Visual Object Classes}
	\acro{R-CNN}{Regional Convolutional Neural Network}
	\acro{ReLU}{Rectified Linear Unit}
	\acro{SSD}{Single Shot MultiBox Detector}
	\acro{YOLO}{You Only Look Once}

\end{acronym}
\include{Inhalt/03_Verzeichnisse/abbildungen}
\setlength{\cftequationsnumwidth}{2,5em}
\setlength{\cftequationsindent}{\cfttabindent}
\addcontentsline{toc}{chapter}{Formelverzeichnis}
\listofequations 
%\listoftables                           % Erzeugen des Tabellenverzeichnisses
\renewcommand{\lstlistlistingname}{Listenverzeichnis}
\lstlistoflistings                      % Erzeugen des Listenverzeichnisses
\setcounter{savepage}{\value{page}}


% ---- Inhalt der Arbeit
\cleardoublepage
\pagenumbering{arabic}                  % Arabische Seitenzahlen für den Hauptteil
\setlength{\parskip}{0.5\baselineskip}  % Abstand zwischen Absätzen
\rmfamily
\renewcommand*{\chapterpagestyle}{scrheadings}
\pagestyle{scrheadings}
\onehalfspacing
\setcounter{chapter}{-1}
\chapter{Vorwort}

Besonderen Dank ist an unseren Betreuer PD Dr. -Ing. Markus Reischl auszusprechen, ohne den die folgenden Forschungsergebnisse nicht zustande gekommen wären. Auch dem Informatik Labor unter Enrico Hühneborg der DHBW ist für die nötige finanzielle Unterstützung zum Erwerb der Drohne zu danken.

\chapter{Grundlagen und Forschungsstand}

Neben einer Einführung in den Anwendungsbereich von \textit{Deep Learning} zur Bildverarbeitung soll sich das folgende Kapitel speziell mit Architekturen unterschiedlicher Objektdetektoren auseinander setzen und herausstellen, wie sich diese voneinander abgrenzen. Davor wird allerdings zunächst grundlegendes Wissen über neuronale Netze und wie diese \glqq lernen\grqq{} vermittelt sowie wie ein eigener Datensatz zu gestalten ist.

\section{Deep Learning zur Bildverarbeitung} \label{bildverarbeitung}

Ein klassisches Anwendungsgebiet von \textit{Deep Learning} zu Bildverarbeitung oder auch allgemein von maschinellem Lernen beschreibt die \textit{Klassifikation}. Hierbei werden bestimmte Kategorien, auch \textit{Klassen} genannt, definiert, in die ein Bild eingeordnet werden soll. Die \textit{Klassifikation} wird anhand von aus dem Bild extrahierten Merkmalen, auch \textit{Features} genannt, getroffen. Die Merkmale werden zu einem \textit{Merkmalsvektor} oder auch \textit{Feature Map} zusammengefasst und von einem \textit{neuronalen Netz} verarbeitet. Das Ergebnis der Verarbeitung durch das neuronale Netz ist die Einordnung in eine bestimme Klasse. 

Zusätzlich zur \textit{Klassifikation} eines Bildes kann das auf dem Bild abgebildete Objekt ebenfalls lokalisiert werden. Es wird dann von sogenannter \textit{Objektdetektion} gesprochen. Es können auch mehrere Objekte auf einem Bild detektiert werden. Ergebnis der Objektdetektion ist somit nicht nur eine Klasseneinordnung sondern ebenso eine klare Positionsangabe des Objektes auf dem Bild. Die Positionsangabe erfolgt durch Angabe einer sogenannten \textit{Bounding Box}. Diese umrahmt das jeweils detektierte Objekt und wird durch ihren linken oberen Eckpunkt sowie ihre Höhe und Breite beschrieben. 

Neben der klassischen \textit{Klassifikation} und der \textit{Objektdetektion} existiert ebenso ein drittes Anwendungsgebiet von \textit{Deep Learning} zu Bildverarbeitung, die \textit{Segmentierung}. Bei der \textit{semantischen Segmentierung} wird versucht, jede einzelne Pixel eines Bildes einer Klasse zuzuordnen und dementsprechend farblich im Bild zu hinterlegen. \textit{Instanzbasierte Segmentierung} hingegen zielt darauf ab, nicht nur jeden Pixel zu einer Klasse zuzuordnen, sondern ebenso eine Identität zu einem Objekt zuzuweisen \cite{RavindraParmar.20180902}. Es setzt sich zusammen aus \textit{semantischer Segmentierung} und paralleler \textit{Objektdetektion}.

Einen Überblick über die vorgestellten Anwendungsgebiete ist in Abbildung \ref{applications} zu sehen.

\begin{figure}[ht]
	\begin{center}
		\includegraphics[width=12cm]{Bilder/applications.png} 
		\caption[Anwendungsgebiete von Deep Learning zur Bildverarbeitung im Überblick]{Anwendungsgebiete von Deep Learning zur Bildverarbeitung im Überblick \cite{PriyaDwivedi.20190328}}
		\label{applications}
	\end{center}
\end{figure}

Für eine einfache \textit{Klassifikation} eines Bildes können einfache sogenannte \textit{Feed-Forward} Netze verwendet werden. Es kann hierbei aber auch auf konventionelle Methoden der Bildverarbeitung zurück gegriffen werden. Für die \textit{Objektdetektion} stehen Architekturen wie \textit{You Only Loom Once} (YOLO), der \textit{Single Shot MultiBox Detector} (SSD) oder neuronale Netze der \textit{Regional Convolutional Neural Networks} (R-CNN) bereit. Aus dieser Familie entstammt ebenso das \textit{Mask R-CNN} Netz, das zur \textit{instanzbasierten Segmentierung} von Objekten verwendet wird.

\section{Problemstellung und Motivation}

Spezielle Beschreibung mit offenen Fragen
\section{Vorgehensweise und Zielsetzung}

Ziele der Arbeit und Strukturierung

\chapter{Grundlagen und Forschungsstand}

Neben einer Einführung in den Anwendungsbereich von \textit{Deep Learning} zur Bildverarbeitung soll sich das folgende Kapitel speziell mit Architekturen unterschiedlicher Objektdetektoren auseinander setzen und herausstellen, wie sich diese voneinander abgrenzen. Davor wird allerdings zunächst grundlegendes Wissen über neuronale Netze und wie diese \glqq lernen\grqq{} vermittelt sowie wie ein eigener Datensatz zu gestalten ist.

\section{Deep Learning zur Bildverarbeitung} \label{bildverarbeitung}

Ein klassisches Anwendungsgebiet von \textit{Deep Learning} zu Bildverarbeitung oder auch allgemein von maschinellem Lernen beschreibt die \textit{Klassifikation}. Hierbei werden bestimmte Kategorien, auch \textit{Klassen} genannt, definiert, in die ein Bild eingeordnet werden soll. Die \textit{Klassifikation} wird anhand von aus dem Bild extrahierten Merkmalen, auch \textit{Features} genannt, getroffen. Die Merkmale werden zu einem \textit{Merkmalsvektor} oder auch \textit{Feature Map} zusammengefasst und von einem \textit{neuronalen Netz} verarbeitet. Das Ergebnis der Verarbeitung durch das neuronale Netz ist die Einordnung in eine bestimme Klasse. 

Zusätzlich zur \textit{Klassifikation} eines Bildes kann das auf dem Bild abgebildete Objekt ebenfalls lokalisiert werden. Es wird dann von sogenannter \textit{Objektdetektion} gesprochen. Es können auch mehrere Objekte auf einem Bild detektiert werden. Ergebnis der Objektdetektion ist somit nicht nur eine Klasseneinordnung sondern ebenso eine klare Positionsangabe des Objektes auf dem Bild. Die Positionsangabe erfolgt durch Angabe einer sogenannten \textit{Bounding Box}. Diese umrahmt das jeweils detektierte Objekt und wird durch ihren linken oberen Eckpunkt sowie ihre Höhe und Breite beschrieben. 

Neben der klassischen \textit{Klassifikation} und der \textit{Objektdetektion} existiert ebenso ein drittes Anwendungsgebiet von \textit{Deep Learning} zu Bildverarbeitung, die \textit{Segmentierung}. Bei der \textit{semantischen Segmentierung} wird versucht, jede einzelne Pixel eines Bildes einer Klasse zuzuordnen und dementsprechend farblich im Bild zu hinterlegen. \textit{Instanzbasierte Segmentierung} hingegen zielt darauf ab, nicht nur jeden Pixel zu einer Klasse zuzuordnen, sondern ebenso eine Identität zu einem Objekt zuzuweisen \cite{RavindraParmar.20180902}. Es setzt sich zusammen aus \textit{semantischer Segmentierung} und paralleler \textit{Objektdetektion}.

Einen Überblick über die vorgestellten Anwendungsgebiete ist in Abbildung \ref{applications} zu sehen.

\begin{figure}[ht]
	\begin{center}
		\includegraphics[width=12cm]{Bilder/applications.png} 
		\caption[Anwendungsgebiete von Deep Learning zur Bildverarbeitung im Überblick]{Anwendungsgebiete von Deep Learning zur Bildverarbeitung im Überblick \cite{PriyaDwivedi.20190328}}
		\label{applications}
	\end{center}
\end{figure}

Für eine einfache \textit{Klassifikation} eines Bildes können einfache sogenannte \textit{Feed-Forward} Netze verwendet werden. Es kann hierbei aber auch auf konventionelle Methoden der Bildverarbeitung zurück gegriffen werden. Für die \textit{Objektdetektion} stehen Architekturen wie \textit{You Only Loom Once} (YOLO), der \textit{Single Shot MultiBox Detector} (SSD) oder neuronale Netze der \textit{Regional Convolutional Neural Networks} (R-CNN) bereit. Aus dieser Familie entstammt ebenso das \textit{Mask R-CNN} Netz, das zur \textit{instanzbasierten Segmentierung} von Objekten verwendet wird.

\section{Neuronale Netze} \label{anns}

Ein neuronales Netz bildet die Grundlage des \textit{Deep Learnings}. Neuronale Netze sind komplexe Datenstrukturen, deren kleinste Einheiten aus sogenannten \textit{Linear Threshold Units} (LTUs) bestehen. Diese geben auf Basis von mehreren Eingangswerten einen durch eine Aktivierungsfunktion beschriebenen Ausgangswert aus. Mehrere LTUs sind zusammen in einer eindimensionalen Schicht angeordnet. Ein oder mehrere solcher Schichten bilden ein Perzeptron, den Grundbaustein eines ANNs. Dabei ist jede LTU einer Schicht mit allen LTUs einer folgenden Schicht verbunden. Hier wird auch von sogenannten vollständig verbundenen Schichten (engl.: \textit{Fully-Connected Layer}) gesprochen. Die Verbindungen sind mit einer Gewichtung versehen, die im Lernprozess angepasst werden und somit für bestimmte Eingangsdaten nur bestimmte LTUs aktivieren. Diejenige LTU die am Ende des Netzes alleinig aktiviert ist gibt die Klassifizierung der Eingangsdaten an. Da allerdings meist nicht nur eine LTU alleinig aktiviert ist, findet die Klassifikation auf Basis von Wahrscheinlichkeiten statt. Hierzu wird meist die \textit{Softmax-Funktion}

\begin{equation} \label{softmax}
h_{w}(x) = \sigma(z)_j = \frac{e^{z_j}}{\sum_{i=0}^n e^{z_i} }
\end{equation}
\equations{Die Softmax-Funktion}

verwendet, die den Wert des $j$-ten LTUs einer Schicht mit allen anderen $n$ Werten der LTUs derselben Schicht ins Verhältnis setzt. Falls das neuronale Netz von der Eingangsschicht zur Ausgangsschicht unidirektional von den Eingangsdaten durchlaufen wird, ohne zu bereits besuchten Schichten zurückzukehren, nennt man das neuronale Netz auch \textit{Feed Forward Network}.

Im Wesentlichen existieren vier Methoden, mit deren Hilfe neuronale Netze trainiert werden können: \textit{Überwachten Lernverfahren}, \textit{Halbüberwachte Lernverfahren}, \textit{unüberwachte Lernverfahren} und das \textit{Reinforcement Learning}. Für die Aufgabe der Objektdetektion sind allerdings wesentlich \textit{überwachten Lernverfahren} von Relevanz, die den Trainingsdaten Lösungen, sogenannte \textit{Labels} oder \textit{Annotations}, hinzufügen. Der Lernprozess wird hierbei über das Gradientenverfahren, basierend auf einer Kostenfunktion, und dem Backpropagation Algorithmus ermöglicht. Die Kostenfunktion ist ein Qualitätsmaß dafür, wie weit die Ausgabe einer LTU vom erwarteten Wert abweicht \cite{AurelienGeron.2018}. Eine bekannte Kostenfunktion für das \textit{überwachte Lernen} ist beispielsweise die \textit{Smooth L1} Funktion

\begin{equation} \label{smooth}
SM_{L1}(\boldsymbol{z},\boldsymbol{o}) = \begin{cases}
\sum_{k=0}^n 0.5(z_k-o_k)^2      & \text{wenn } |x| < 1\\
\sum_{k=0}^n |z_k-o_k| - 0.5   & \text{sonst}
\end{cases}
\end{equation}
\equations{Die Smooth L1 Funktion}

, bei der $z$ der erwartete Ausgabevektor des Perzeptrons ist, während $o$ die momentane Ausgabe darstellt.

Für weitere Informationen über das Perzeptron, Lernmethoden, Gradientenverfahren und Backpropagation sind tiefgehendere Kapitel im Anhang hinterlegt.


\section{Hyperparameter}

Hyperparameter sind die Parameter, die zur anfänglichen Konfiguration des neuronalen Netzes als auch zur Konfiguration des Lernprozesses heran gezogen werden. Um im Laufe der Arbeit verstehen zu können, wie die Objektdetektoren auf Seiten der Netzarchitektur und des Lernverhaltens optimiert wurden, ist demnach ein kurzer Einblick in den Themenbereich der Hyperparameter von Nöten.

\subsection*{Anzahl der LTUs}
Die Anzahl der LTUs im ANN ist dafür ausschlaggebend, wie hoch der Komplexitätsanspruch eines Klassifizierungsproblems sein darf, um noch vom ANN gelöst werden zu können. Die Anzahl der LTUs hängt hauptsächlich von den Eingangsdaten ab. Über die optimalste Anzahl an LTUs pro Schicht lässt sich allerdings nur schwer etwas vorhersagen. Generell gilt, dass bei gleicher Anzahl an LTUs tiefere Netze eine weitaus höheren Parametereffizienz aufweisen als breitere Netze, da diese schneller gegen den gewünschten Zustand konvergieren. Zudem lassen sie sich somit schneller und kostengünstiger trainieren. So müssten bei einem 2x32 Netz 1024 Gewichtungen angepasst werden, während es bei einem 32x2 Netz dies nur 128 sind \cite{AurelienGeron.2018}.

\subsection*{Initialisierung der Gewichtungen}
Auch stellt die Initialisierung der Gewichte eines ANNs zu Beginn des Trainingsprozesses eine berechtigte Frage dar. Falls keine bereits trainierten ANNs für ein Klassifikationsproblem vorliegen, so werden die Gewichtungen meist zufällig nach einer Normalverteilung gewählt \cite{AurelienGeron.2018}. 

Dies hat allerdings zur Folge, dass nach der Berechnung der gewichteten Summen aller LTUs die Gewichtungswerte der folgenden Schicht nicht mehr normalverteilt sind, da für die Varianz zweier unkorrelierter Zufallsvariablen das Superpositionsprinzip

\begin{equation} \label{varianz}
Var(X + Y) = Var(X) + Var(Y)
\end{equation}
\equations{Superpositionsprinzip anhand der Varianz}

gilt.

Durch die größer werdende Standardabweichung können demnach Gewichtungswerte entstehen, die weit vom Mittelwert Null abweichen. Dies kann wiederum dazu führen, dass der Gradientenabstieg während des Backpropagation-Verfahrens nur langsam vollzogen werden kann, da der Gradient bei bestimmten Aktivierungsfunktionen (siehe Abbildung \ref{sigmoid}) gegen Null konvergiert \cite{AurelienGeron.2018}. 

Eine \textit{Xavier Initialisierung} umgeht das Problem der sogenannten \textit{schwindenden Gradienten}, indem die Gewichte nach

\begin{equation} \label{xavier}
W \sim U[-\frac{\sqrt{6}}{\sqrt{n_{j} + n_{j+1}}},\frac{\sqrt{6}}{\sqrt{n_{j} + n_{j+1}}}]
\end{equation}
\equations{Standardverteilung nach Xavier Initialisierung}

gleichverteilt werden, wobei $n{j}$ die Anzahl an LTUs der $j-ten$ Schicht sind \cite{XavierGlorot.2010}.

\subsection*{Anzahl an Epochen}

Die Anzahl der Epochen beschreibt die Durchläufe durch einen bestimmten Trainingsdatensatz während der Trainingsphase. Ist die Anzahl zu hoch gewählt, wird Gefahr gelaufen, sogenanntes \textit{Overfitting} des ANNs zu erreichen. Dies bedeutet ein fehlendes Abstraktionsvermögen des ANNs und damit alleinig eine richtige Erkennung der Trainingsdatensätze zu ermöglichen.  

\subsection*{Lernrate}

Die Lernrate $\eta$ (\ref{gradientenverfahren}) gibt an, wie groß die Sprünge sein sollen und damit indirekt wie viele Iterationen benötigt werden, um das Minimum der Kostenfunktion zu erreichen. Ziel der Anpassung einer Lernrate ist es, mit möglichst wenig Iterationen und Testdaten die optimale Konstellation des neuronalen Netzes zu berechnen. Deshalb wird sie standardmäßig zu Beginn der Iterationen groß gewählt, um sich dem Minimum schnell zu nähern, während sie am Ende immer kleiner gewählt wird, um nicht über das globale Minimum hinaus zu gehen. Dieses Vorgehen wird als \textit{Simulated Annealing} bezeichnet, während das Funktion zum Festlegen der Lernrate als \textit{Learning Schedule} betitelt wird \cite{AurelienGeron.2018}.

Die Anzahl der Durchläufe wird zu Beginn des Verfahrens zunächst hoch angesetzt, das Verfahren wird aber genau dann gestoppt, sobald der Gradientenvektor unter eine gewisse Abbruchgrenze fällt. Zwar ist das globale Minimum zu diesem Zeitpunkt noch nicht erreicht, allerdings kann es auch nie vollkommen erreicht werden, da die für das Gradientenverfahren genutzten Aktivierungsfunktionen nie einen partiellen Ableitungswert gleich Null zulassen \cite{AurelienGeron.2018}. In diesem Sinne wird auch von \textit{Toleranz} gesprochen.

\subsection*{Moment}

Das Gradientenverfahren kann beschleunigt werden, indem während des Gradientenabstiegs frühere Gradienten Einfluss auf den nächsten Gradientenschritt nehmen. Es wird ein \glqq Momentum\grqq{} aufgebaut. Damit das Momentum 

\begin{equation} \label{momentum}
\begin{split}
m_x = \beta \cdot m_{x-1} + \eta\frac{\partial E}{\partial w_{ij}} \\
w_{ijt} = w_{ijt-1} - m 
\end{split}
\end{equation}
\equations{Momentum Optimierung}

allerdings nicht zu groß wird, beschränkt der Hyperparameter $\beta \in [0,1]$ die Größe des Momentums \cite{AurelienGeron.2018}.

Die Momentum Optimierung kann dazu benutzt werden, das \textit{stochastische Gradientenverfahren} bzw. \textit{Mini-Batch} Verfahren zu beschleunigen und lokale Minima besser zu überwinden.

\subsection*{Auswahl des Gradientenverfahrens}

Generell wird zwischen drei verschiedenen Arten unterschieden, das Gradientenverfahren durchzuführen (siehe Abbildung \ref{gradient}):

\begin{figure}[ht]
	\begin{center}
		\includegraphics[width=15cm]{Bilder/gradient_descent.png} 
		\caption[Gradientenverfahren]{Gradientenverfahren \cite{ImadDabbura.20171221}}
		\label{gradient}
	\end{center}
\end{figure}

Beim \textit{Batch} Verfahren werden in einem Trainingsdurchlauf, der \textit{Epoche}, alle vorhandenen Daten des Trainingsdatensatzes herangezogen, um einen Gradientenabstieg zu vollziehen. Dies ist bei großen Trainingsdatensätzen auffällig langsam, dafür aber hinsichtlich der Erreichung des lokalen Minimums sehr zielstrebig \cite{AurelienGeron.2018}.

Das \textit{stochastische Gradientenverfahren} führt nach jedem einzelnen Dateneintrag im Trainingsdatensatz einen Gradientenabstieg durch. Da nur wenige Daten des ANNs verändert werden müssen, ist dieses Verfahren deutlich schneller, dafür aber unregelmäßiger hinsichtlich der Erreichung des Minimums. Oft wird das stochastische Gradientenverfahren verwendet, wenn nicht der komplette Trainingsdatensatz in den Hauptspeicher oder Grafikspeicher geladen werden kann. Diese Fähigkeit wird oft als \textit{Out-of-Core} Fähigkeit bezeichnet. Es hat auch den Vorteil, besser das globale Minimum der Kostenfunktion aufzufinden, da bei lokalen Minima die Chance besteht, durch den unregelmäßigen Gradientenabstieg das lokale Minimum wieder zu überwinden \cite{AurelienGeron.2018}.

Ein Kompromiss der beiden Verfahren bietet das \textit{Mini-Batch} Verfahren, bei dem wiederholt Teilmengen des gesamten Datensatzes für einen Gradientenabstieg verwendet werden. Genauso wie das \textit{Batch} Verfahren bietet das \textit{Mini-Batch} Verfahren den Vorteil, die partiellen Ableitungen als Matrizenoperationen auf die Grafikkarten auszulagern, um die Performanz durch Parallelisierung zu steigern \cite{AurelienGeron.2018}.

\subsection*{Aktivierungsfunktionen}

Zwei bekannte und ähnliche Aktivierungsfunktionen sind die \textit{Sigmoid-Funktion} und die \textit{Tangens Hyperbolicus} Funktion. Da diese allerdings durch ihr schnelles Konvergieren gegen den Grenzwert anfällig für das Problem \textit{schwindender Gradienten} sind \cite{AurelienGeron.2018}, wird die \textit{Rectified Linear Unit} (ReLU) bzw. \textit{Parametric/Leaky Rectified Linear Unit} (PReLU/LRelU) Aktivierungsfunktion bevorzugt (siehe Abbildung \ref{relu}). 

\begin{figure}[ht]
	\subfigure[RELU]{\includegraphics[width=7.5cm]{Bilder/relu.png}} 
	\subfigure[PReLU/LReLU]{\includegraphics[width=7.5cm]{Bilder/prelu.png}} 
	\caption[ReLU-Aktivierungsfunktionen]{ReLU-Aktivierungsfunktionen \cite{DanqingLiu.20171130}} 
	\label{relu}
\end{figure} 

Bei ReLU kann es während des Trainingsprozesses dazu kommen, dass LTUs nach dem Gradientenabstieg einen negativen Wert aufweisen, weshalb sie nicht weiter aktiviert werden und für den Rest der Trainingsdauer \glqq tot\grqq{} sind. Um dies zu verhindern, wurde \textit{LReLU} dazu genutzt, um eine Reaktivierung zu ermöglichen, da auch für negative LTU Werte ein Gradient der Aktivierungsfunktion bestimmt werden kann. Bei \textit{LReLU} ist die Steigung der Funktion im zweiten Quadranten statisch gewählt, während sie bei \textit{PReLU} dynamisch von neuronalen Netz während des Trainingsprozesses selbst gelernt werden kann \cite{AurelienGeron.2018}.

Eine letzte Variante der Aktivierungsfunktionen beschreibt die \textit{ELU} Funktion (siehe Abbildung \ref{elu}).

\begin{figure}[ht]
	\begin{center}
		\includegraphics[width=12cm]{Bilder/elu.png} 
		\caption[ELU]{ELU \cite{DanqingLiu.20171130}}
		\label{elu}
	\end{center}
\end{figure}

Sie besitzt nicht nur die Eigenschaft schwindende Gradienten und damit nicht anpassbare, sogenannte \glqq tote\grqq LTUs zu verhindern, sondern ist im gesamten Definitionsbereich ebenso eine stetig differenzierbare Funktion, was das Gradientenverfahren beschleunigt. Als Standardwert für den Streckungsfaktor $\alpha$ der niederen Funktion wird oft Eins verwendet. Nachteil der \textit{ELU} Funktion ist der erhöhte Rechenaufwand, was aber durch die schnellere Konvergenz kompensiert wird \cite{AurelienGeron.2018}.


\section{Datensatzlehre}

\subsection*{Datensatzformate}

Basierend darauf, welcher Objektdetektor trainiert werden soll, muss der zum Training verwendete Datensatz in einem bestimmten Format vorliegen. Zum Trainieren des \textit{SSDs} wird das sogenannte \textit{Pascal Visual Object Classes} (PascalVOC) Format benötigt. 

Es definiert eine Unterteilung in \textit{Annotations}, \textit{ImageSets} und \textit{JPEGImages}. Während in dem Ordner \textit{JPEGImages} alle Bilder des Datensatzes vorhanden sind, befindet sich unter anderem die Information über die vorhandenen Objekte in dem Bild im Ordner \textit{Annotations}. Für jedes Bild des Datensatzes werden die Informationen in einer gleichnamigen XML-Datei abgelegt (siehe Listing \ref{pascalvoc:PascalVOC}).

\lstset{language=XML}
\lstinputlisting[
label=pascalvoc:PascalVOC,
caption=PascalVOC Bildannotation,
captionpos=b,
firstline=1,
lastline=26
]{Quellcode/annotation.xml}

Neben allgemeinen Metainformationen über das Bild befindet sich hier ebenso eine Liste aller markierten Objekte. Pro Objekt wird die Klassifikationskategorie, die Ausrichtung (z.B. \glqq Frontal\grqq{}), die Information über vollständiges Erscheinen im Bild, die Information über schwere Erkennbarkeit und die Bounding Box angegeben. Im Ordner \textit{ImageSets/Main} wird eine Unterteilung in Trainings- und Testdatensatz durch zwei Textdateien realisiert, die die Dateinamen der Bilddateien als Auflistung enthalten \cite{RenuKhandelwal.2019}. 

Das \textit{YOLO} Format für den \textit{YOLO} Objektdetektor definiert in einer \textit{.names}-Datei alle im Datensatz vorhandenen Kategorien durch simple Auflistung der Bezeichner. Die Bilder werden zusammen mit ihren Annotationen in einem separaten Ordner abgelegt. Die Annotationen folgen hier dem Format:

$<Kategorie-ID>\:<Zentrum-X>\:<Zentrum-Y>\:<Breite>\:<Hoehe>$

Die Unterteilung in Trainings- und Testdatensatz erfolgt durch Referenzierung der Bildpfade in zwei getrennten Textdateien. Schließlich wird in einer \textit{.data}-Datei der Pfad zu den beiden Textdateien und zur \textit{.names}-Datei sowie die Anzahl an Kategorien gespeichert \cite{ArunPonnusamy.20191006}. 

\subsection*{Datensatzzusammensetzung}

Zum Erstellen und Auswählen eines \textit{Deep Learning} Modells wird der Datenbestand in der Regel in drei Kategorien unterteilt. Ein Datensatz wird für das Training des Modells verwendet. Durch das anschließende Anwenden des Modells auf zuvor ungesehene Daten, den Testdaten, wird der \textit{Verallgemeinerungsfehler} gemessen, der möglichst niedrig ausfallen sollte. Fällt der allgemeine Fehler während des Trainings niedrig aus, der \textit{Verallgemeinerungsfehler} während des Testdurchlaufs allerdings hoch, so liegt klassisches \textit{Overfitting} vor, die Trainingsdaten wurden auswendig gelernt \cite{AurelienGeron.2018}. 

Anschließend werden in mehreren Durchläufen die Hyperparameter des Trainingsprozesses angepasst, sodass letztendlich der \textit{Verallgemeinerungsfehler} für die Testdaten niedrig ausfällt. Kommt es anschließend zum Einsatz des Modells in der Produktivumgebung, so können trotz allem unerwartete Ergebnisse bezüglich des Abstraktionsvermögens des Modells auftreten. Dies liegt daran, dass das Modell allein auf die Testdaten hin optimiert wurde. Um dies zu vermeiden wird ein dritter Datensatz, der Validierungsdatensatz, eingeführt. Mehrere Modelle werden dabei durch den Validierungsdatensatz getestet und das dabei am besten abschneidende Modell mit dessen Hyperparametern ausgewählt. Der eigentliche Testdatendatz wird anschließend nur noch zur Abschätzung des \textit{Verallgemeinerungsfehlers} verwendet \cite{AurelienGeron.2018}. 

Oft wird der Trainingsdatensatz mit dem Validierungsdatensatz zum sogenannten \textit{Trainval} Datensatz zusammengeführt. Dies steht im Kontext des sogenannten \textit{K-Kreuzvalidier-""ungsverfahrens}. Dabei wird der \textit{Trainval} Datensatz in K gleich große, komplementäre Untermengen unterteilt. Eine dieser Untermengen dient anschließend als Validierungsdatensatz. Für jedes zu trainierende Modell mit unterschiedlichen Hyperparametern wird eine andere Untermenge als Validierungsdatensatz ausgewählt. Hierdurch steigt die Aussagekraft des Abstraktionsvermögens nach der Validierung und zudem müssen keine Trainingsdaten dauerhaft für die Validierung zurück gelegt werden. In der Regel werden 80\% der Gesamtdaten als \textit{Trainval} Datensatz verwendet \cite{AurelienGeron.2018}.

\subsection*{Qualität und Quantität der Daten}

Um ein funktionsfähiges Modell zu trainieren, muss der Datensatz einem gewissen Standard nachkommen. Demnach müssen die zu klassifizierenden Objekte vollständig im Bild enthalten und gut erkennbar sein. Zwar gibt es gerade im \textit{PascalVOC} Datensatzformat ebenso die Möglichkeit, Objekte als \glqq schwierig erkennbar\grqq{} zu markieren, dennoch sollen solche Objekte nicht die Mehrheit im gesamten Datensatz ausmachen. Auch die Aufnahme von Objekten in unterschiedlichen Umgebungen, Entfernungen und Blicklagen fördert langfristig das Abstraktionsvermögen des Modells. 

Ebenso muss ein ausreichend großer Datensatz vorliegen, um das gewünschte Abstraktionsvermögen des Modells zu erreichen. Die Ergebnisse aus \ref{result} wurden beispielsweise durch Kombination der \textit{Trainval} Datensätze von PascalVOC 2007 und 2012 erzielt und umfasst 16.551 Bilder im Trainingsverfahren \cite{ssd.20161229} \cite{MarkEveringham.20070607} \cite{MarkEveringham.20120521}. 

Unter Hinzunahme des COCO \textit{trainval135k} Datensatzes erreicht der \textit{SSD} sogar das beste Ergebnis aus der ursprünglichen Veröffentlichung mit einer durchschnittlichen Präzision von 81.6\% \cite{ssd.20161229}. 

\subsection*{Techniken zum Trainieren bei geringen Datenmengen}

Bei Betrachtung der obigen Ergebnisse wird schnell deutlich, dass für ein komplexeres \textit{Deep Learning} Modell ein umfangreicher Datensatz von Nöten ist. Allerdings gibt es zwei bekannte Techniken, wie auch mit kleineren Datenbeständen ein sehenswertes Ergebnis erzielt werden kann. 

Beim sogenannten \textit{Transfer Learning} können von einem bereits für ein ähnliches Problem trainiertes Modell die ersten Schichten des neuronalen Netzes für das neue Modell wiederverwendet werden. Die übernommenen Gewichtungen werden nicht mit trainiert. Neben einer kleineren Datenmenge zum Trainieren hat das \textit{Transfer Learning} ebenso den Vorteil das Training selbst zu beschleunigen \cite{AurelienGeron.2018}.

Eine weitere Technik beschreibt das künstliche Vergrößern des Datensatzes durch affine Transformationen wie Translation, Rotation oder Skalierung und wird \textit{Data Augmentation} genannt \cite{AurelienGeron.2018}.

\section{Objektdetektoren}

\subsection{Convolutional Neural Networks}

Auch kann bei tiefen ANNs eine gewisse Klassifikationshierarchie für jede Schicht zugeordnet werden. Bei der Bilderkennung sind beispielsweise die ersten verborgenen Schichten dafür zuständig kleinere Muster zu erkennen, während mit fortschreitenden Schichten diese Muster zu immer größeren Mustern zusammengefasst werden können. \cite[S. 271 f.]{AurelienGeron.2018}

Bei einem 28x28 Pixel großen schwarz-weiß Bild sind beispielsweise für die Modellierung des Grauwertes jedes Pixels insgesamt 784 Input LTUs notwendig. Da die erkannten Strukturen von vielen detaillierten nach und nach zu wenigen verallgemeinerten zusammengefasst werden können, sinkt die Anzahl an benötigten LTUs pro Schichten in einem Feed Forward Netz zur Ausgabeschicht. Ein trichterförmiger Aufbau des ANNs ist somit nicht unüblich. \cite[S. 271 f.]{AurelienGeron.2018}

Diese Architektur ermöglicht ebenso die Wiederverwendbarkeit einzelner Schichten und Gewichtungen für ähnliche Klassifikationsprobleme, bei denen gleiche Muster vorzufinden sind. \cite[S. 271]{AurelienGeron.2018}
\subsection*{Mean Average Precision}

Um die Genauigkeit von Objektdetektoren zu messen, wird oft die Metrik \textit{mean Average Precision} (mAP) gewählt. Diese setzt sich aus zwei grundlegenden Größen zusammen:

\begin{equation} \label{precisionandrecall}
\begin{split}
Precision = \frac{True Positive}{True Positive + False Positive} \\
\\
Recall = \frac{True Positive}{True Positive + False Negative}
\end{split}
\end{equation}
\equations{Precision und Recall}

\textit{Precision} sagt also etwas über die Verlässlichkeit einer Klassifikation aus, während \textit{Recall} Aussagen über die Erkennungsfähigkeit eines Objektdetektors trifft. Wichtig ist es hierbei anzumerken, dass mehrfach detektierte Objekte nur einmal als positiver Befund aufgefasst werden, die restlichen Detektionen gehen als \textit{False Positives} ein \cite{PaulHenderson.2017}.

\begin{figure}[H]
	\begin{center}
		\includegraphics[width=8cm]{Bilder/iou_equation.png} 
		\caption[Intersection over Union]{Intersection over Union \cite{AdrianRosebrock.20161107}}
		\label{iou}
	\end{center}
\end{figure}

Die Tatsache, ob eine Bounding Box das gewünschte Objekt enthält und demnach ein positiver Fall vorliegt, wird anhand des \textit{confidence scores} bestimmt. Er berechnet sich aus der Multiplikation der Wahrscheinlichkeit für eine Klasse mit der sogenannten \textit{Intersection over Union} (IoU) (siehe Abbildung \ref{iou}) der jeweiligen ausgewählten Bounding Box. Die \textit{IoU} beschreibt ein Maß der Überdeckung der detektierten Bounding Box zur wahren Bounding Box. Der \textit{confidence score} sagt also etwas über die Gewissheit der Klassifikation aus. Für den kompletten Datensatz werden nun für unterschiedliche \textit{confidence scores} jeweils \textit{Precision} und \textit{Recall} bestimmt und anschließend in einem Graphen aufgetragen. Meistens werden die \textit{confidence scores} so gewählt, sodass sich eine äquidistante Abstufung in den \textit{Recall} Werten ergibt \cite{DingfuZhouJinFangXibinSongChenyeGuanJunboYinYuchaoDaiRuigangYang.2019}. 

\begin{figure}[H]
	\subfigure{\includegraphics[width=12cm]{Bilder/map_graph1.png}} 
	\subfigure{\includegraphics[width=12cm]{Bilder/map_graph2.png}} 
	\caption[Berechnung mAP]{Berechnung mAP \cite{JonathanHui.20180307}} 
	\label{map}
\end{figure} 

Im Graphen ist meist ein klassisches \glqq Zick-Zack\grqq{} Muster zu erkennen (siehe Abbildung \ref{map}). Dieses Muster wird geglättet, indem nach jedem Einbruch für jeden \textit{Recall} Wert der maximale \textit{Precision} Wert rechts des aktuellen \textit{Recalls} übernommen wird. Wird anschließend das diskrete Integral über alle \textit{Recall} Werte gebildet, so ergibt sich der \textit{Average Precision} Wert für eine zu klassifizierende Kategorie. Der Mittelwert der  \textit{Average Precisions} über alle Klassifikationskategorien hinweg ergibt letztendlich den \textit{mAP} Wert \cite{JonathanHui.20180307}. 

\section{Objektdetektoren} \label{detection}

Ein Teilziel der Machbarkeitsstudie ist es, anhand vorbestimmter Kriterien eine ausgewählte Menge von Objektdetektoren zu vergleichen. Um im Laufe der Arbeit zu verstehen, wie diese Auswahl zu Stande kommt und wie sich bestimmte Ergebnisse im Vergleich begründen lassen, ist eine Einführung in die unterschiedlichen Architekturen der Objektdetektoren unumgänglich. 

\subsection*{Regional Convolutional Neural Networks}

\textit{Regional Convolutional Neural Networks} (R-CNNs) vertreten den Ansatz, für ein Bild mehrere Lokationsvorschläge für mögliche Objekte zu liefern, sogenannte \textit{Regions of Interest} (RoIs), und diese anschließend zu klassifizieren. 

\begin{figure}[H]
	\begin{center}
		\includegraphics[width=8cm]{Bilder/rcnn.png} 
		\caption[R-CNN Architektur]{R-CNN Architektur \cite{RohithGandhi.20180709}}
		\label{rcnn}
	\end{center}
\end{figure}

Bei dem klassischen \textit{R-CNN} Detektor werden durch den \textit{Selective Search} Algorithmus 2000 solcher \textit{RoIs} vorgeschlagen. Zur Merkmalsextraktion wird für jede \textit{RoI} anschließend ein CNN eingesetzt. Der resultierende \textit{Feature Vektor} wird zur Klassifikation eines Objektes einer \textit{Support Vector Machine} (SVM) unterzogen. Um zusätzlich die Bounding Boxen akkurat zu bestimmen, wird der \textit{Feature Vektor} zudem einem \textit{Bounding Box Regressor} unterzogen (siehe Abbildung \ref{rcnn}) \cite{RossGirshickJeffDonahueTrevorDarrellJitendraMalik.2016}. 

Da der sogenannte \textit{Region Proposal} Schritt durch den \textit{Selective Search} Algorithmus allerdings viel Zeit in Anspruch nimmt, entstand eine Weiterentwicklung des \textit{R-CNN } Netzes, das \textit{Fast R-CNN} Netz. Dieses tauscht den Schritt des \textit{Selective Search} Algorithmus mit dem Einsatz des CNNs. Außerdem wird das klassische CNN leicht angepasst. Bei \textit{Fast R-CNN} wird ein Bild zunächst einem CNN unterworfen. Bevor eine \textit{Feature Map} durch \textit{Fully-Connected Layer} zu einem einzigen \textit{Feature Vektor} vereinfacht wird, werden aus der \textit{Feature Map} die verschiedenen \textit{RoIs} extrahiert. Dies geschieht wiederrum mit dem \textit{Selective Search} Algorithmus, mit dem Unterschied, dass dieser nun nur auf der \textit{Feature Map} operiert und nicht auf dem gesamten Bild. Durch \textit{RoI Pooling Layer} werden die einzelnen entstandenen Regionen in eine feste Größe transformiert und einzeln einer Klassifikation durch \textit{Fully-Connected Layer} und einer Softmax-Funktion unterworfen. Die Komponente mit dem \textit{Bounding Box Regressor} bleibt gleich. Durch den Tausch des CNNs mit dem \textit{Selective Search} Algorithmus werden die mathematischen Faltungsoperationen nur einmal statt 2000 Mal pro Bild ausgeführt, was die Performanz des Detektors gegenüber eines klassischen \textit{R-CNNs} enorm steigert \cite{RossGirshick.2015}.

\begin{figure}[H]
	\begin{center}
		\includegraphics[width=8cm]{Bilder/fasterrcnn.png} 
		\caption[Faster R-CNN Architektur]{Faster R-CNN Architektur \cite{RohithGandhi.20180709}}
		\label{fasterrcnn}
	\end{center}
\end{figure}

Die letzte Optimierung der \textit{R-CNN} Familie entstand durch das \textit{Faster R-CNN} Netz. Dieses ersetzt den statischen \textit{Selective Search} Algorithmus des \textit{Fast R-CNN} Detektors durch ein eigenes lernfähiges, sogenanntes \textit{Region Proposal Network} (RPN)  (siehe Abbildung \ref{fasterrcnn}) \cite{ShaoqingRenKaimingHeRossGirshickJianSun.2016}.

Neben dem Einsatz von \textit{R-CNN} Detektoren zur Objektdetektion existiert ebenso ein Ansatz zur instanzbasierten Segmentierung, das \textit{Mask R-CNN} Netz. Es nimmt zwei wichtige Anpassungen an der Architektur des \textit{Faster R-CNN} Netzes vor. Da bei Segmentierungsproblemen eine genauere Abgrenzung von Objekt und Hintergrund notwendig ist, wird das \textit{RoI Pooling Layer} durch ein \textit{RoI Align Layer} ausgetauscht. Hierbei wird das Rundungsproblem beim Pooling behoben. Angenommen eine \textit{RoI} von 16x16 Pixeln wird mit einem \textit{MEAN-Pooling Layer} der Schrittweite Drei verarbeitet, so ergibt sich pro Pooling Schritt ein Einzugsgebiet von 5.33 Pixeln. Dieses wurde abgerundet auf 5 Pixel. Bei \textit{RoI Align Layern} wird durch bilineare Interpolation der Wert des 5.33ten Pixels ermittelt und in das Pooling mit einbezogen. Dies ermöglicht eine genauere Segmentierung an den Grenzen eines Objektes.

Außerdem wird parallel zum RPN ein sogenanntes \textit{Fully-Convolutional Network} (FCN) eingesetzt, einem Netz, dass rein aus \textit{Convolutional Layern} besteht. Es dient, um für jede existierende Klasse eine pixelbasierte binäre Maske auszugeben, die für jeden Pixel die Zugehörigkeit zu einer Klasse bestimmt. Basierend auf dieser Maske werden die detektierten Objekte anschließend farblich hervorgehoben \cite{KaimingHeGeorgiaGkioxariPiotrDollarRossGirshick.20180224}.
\subsection*{Single Shot MultiBox Detector}

Zwar liefern die oben genannten Objektdetektoren akkurate Ergebnisse, allerdings sind sie als zu rechenintensiv und langsam einzuordnen, als dass sie für Echtzeit Applikationen eingesetzt werden könnten. Der \textit{Single Shot MultiBox Detector} (SSD) unterscheidet sich von vorhergehenden Modellen, wie beispielsweise den \textit{R-CNN} Detektoren, dahingehend, dass er bewusst auf den Schritt der Generierung von Bounding Box Vorschlägen und des Poolings verzichtet, um wesentlich schneller ablaufen zu können als andere Objektdetektoren. Die Präzision der Klassifikationen bleibt hierbei erhalten, selbst Bilder niedriger Auflösung können weiterhin verarbeitet werden. Dem \textit{SSD} genügt also ein einziges tiefes neuronales Netz zum Lokalisieren und Klassifizieren von Objekten. 

\newpage

Wie der \textit{SSD} aufgebaut ist und welche Ansätze er verfolgt, soll in diesem Unterkapitel erläutert werden \cite{ssd.20161229}. 

\begin{figure}[H]
	\begin{center}
		\includegraphics[width=15cm]{Bilder/ssd_framework.png} 
		\caption[SSD Bounding Box Vorschläge]{SSD Bounding Box Vorschläge - Objekte unterschiedlicher Skalierung, hier Katze (blau) und Hund (rot), werden durch unterschiedlich große Gittereinteilungen und Bounding Box Seitenverhältnissen detektiert \cite{ssd.20161229}.}
		\label{framework}
	\end{center}
\end{figure}

Die Architektur des \textit{SSD} zielt darauf ab, durch unterschiedlich große \textit{Convolutional Layer} \textit{Feature Maps} unterschiedlicher Skalierung in die Klassifikation mit einfließen zu lassen. Anschaulich kann es sich vorgestellt werden, als werde das Bild in mehrere unterschiedlich große Gitterstrukturen unterteilt und die resultierenden Zellen jeweils einzeln klassifiziert. Dadurch ist es möglich, Objekte unterschiedlicher Größe zu erkennen. Für jede Zelle im Gitter wird eine gleiche Anzahl vordefinierter Bounding Boxen, die unterschiedliche Seitenverhältnisse aufweisen, definiert. Daher entstammt der Name \glqq MultiBox\grqq{}. Abbildung \ref{framework} zeigt beispielsweise, wie eine Katze (in blau) und ein im Vergleich zur Katze größerer Hund (in rot) durch unterschiedlich große Gittereinteilungen und Bounding Box Seitenverhältnisse detektiert werden. Durch die \textit{MultiBox} Eigenschaft wird ebenso sichergestellt, dass sowohl horizontal als auch vertikal ausgeprägte Objekte in der selben Zelle gleichzeitig erkannt werden können (siehe Abbildung \ref{boundingboxes}) \cite{ssd.20161229}.

\begin{figure}[H]
	\begin{center}
		\includegraphics[width=7cm]{Bilder/bounding_boxes.png} 
		\caption[Bounding Boxes]{Bounding Boxes - In einer Gitterzelle treten Bounding Box Vorschläge mehrerer unterschiedlicher Skalierungen auf, um auch zwei Objekte unterschiedlicher Skalierung, deren Zentrum in der selben Gitterzelle liegt, gleichzeitig detektieren zu können \cite{AndrewNg.2019}.}
		\label{boundingboxes}
	\end{center}
\end{figure}

Für jede dieser Bounding Boxen bestimmt der \textit{SSD} Wahrscheinlichkeiten für Klassenzugehörigkeiten als auch Verschiebungen der vordefinierten Bounding Box zur wahren Bounding Box des Objekts für jede Klasse. Die Kostenfunktion ist durch die gewichtete Summe des Lokalisationsverlustes und des Klassifikationsverlustes bestimmt. Während der Klassifikationverlust durch eine Softmax-Funktion (\ref{softmax}) bestimmt werden kann, wird der Lokalisationsverlust über die Smooth L1 Funktion (\ref{smooth}) bestimmt \cite{ssd.20161229}.

\begin{figure}[H]
	\begin{center}
		\includegraphics[width=15cm]{Bilder/ssd_architecture.png} 
		\caption[SSD Architektur]{SSD Architektur \cite{ssd.20161229}}
		\label{architecture}
	\end{center}
\end{figure}

Technisch basiert der \textit{SSD} auf der Idee eines \textit{Feed-Forward Convolutional Networks} (siehe Abbildung \ref{architecture}). Er benutzt nach dem Prinzip des \textit{Transfer Learnings} ein \textit{VGG-16} Basis Netzwerk\footnote{\textit{VGG-16} ist ein auf dem Datensatz von \textit{ImageNet} basierendes neuronales Netz, das bis zu 1000 unterschiedliche Kategorien klassifizieren kann \cite{KarenSimonyan.2015}.}, dessen \textit{Fully-Connected Layer} am Ende entfernt wurden. Die resultierende \textit{Feature Map} wird nun einer Reihe von ständig kleiner werdenden \textit{Convolutional Layern} unterzogen. Jedes \textit{Convolutional Layer} kann eine feste Anzahl an Detektionen bestimmen. Eine Detektion wird durch eine Klassenangabe und die Lage einer vorhergesagten Bounding Box bestimmt. Eine Bounding Box wird wie bereits erläutert durch den linken oberen Eckpunkt $P(x,y)$ und eine Höhe und Breite bestimmt. Bei $c$ Klassen hat der \textit{Feature Vektor} einer Detektion demnach die Größe $c+4$. Bei einer \textit{Feature Map} Größe von $mxn$ und $k$ verschiedenen vordefinierten Bounding Boxen ergeben sich also $m \cdot n \cdot k \cdot (c+4)$ verschiedene \textit{Feature Vektoren} für eine \textit{Feature Map} \cite{ssd.20161229}. Diese \textit{Feature Vektoren} werden nun an das Ende des Netzes zur Klassifikation weitergeleitet.

Dieser Vorgang wird für alle \textit{Feature Maps} für alle \textit{Convolutional Layer} durchgeführt. Die daraus folgende Menge an Detektionen wird durch ein \textit{Non Maximum Suppression Layer} in ihrer Größe reduziert. Als Maß zur Filterung wird die \textit{IoU} der detektierten Bounding Box zur wahren Bounding Box verwendet. Überschreitet diese einen Wert von 0.5, so ist diese der originalen Bounding Box zugeordnet. Demnach ist es auch möglich, dass eine originale Bounding Box mehreren vordefinierten Bounding Boxen zugeordnet werden kann \cite{ssd.20161229}.

Während des Trainingsprozesses des \textit{SSD300}\footnote{SSD300 verwendet Bilder der Auflösung 300x300 Pixel. Alternativ existiert ebenso SSD512 für Bilder der Auflösung 512x512 Pixel. Die Bilder können jedoch auch kleiner als die vorgegebene Auflösung gewählt werden.} wurde eine Lernrate von $\eta = 10^{-3}$ für das \textit{Mini-Batch} Verfahren mit Batchgröße 32 und Moment $\beta = 0.9$ verwendet. Die Gewichtungen wurden \textit{Xavier} initialisiert. Nach 40.000 Iterationen wurde die Lernrate für 10.000 Iterationen auf $\eta = 10^{-4}$ reduziert und schließlich auf $\eta = 10^{-5}$ \cite{ssd.20161229}. Auf Basis der \textit{PascalVOC} Datensätze aus 2007 und 2012 wurde mit dieser Konfiguration eine \textit{mAP} von 74.3\% für \textit{SSD300} respektive 76.8\% für \textit{SSD512} erreicht.

\subsection{You Only Look Once}

Der Algorithmus \textit{You Only Look Once} (YOLO) ist ein weiterer Objekterkennungsalgorithmus und betrachtet statt separaten Bildregionen das komplette Bild. Er benutzt nur ein neuronales Netz, um Bounding Boxen und Wahrscheinlichkeiten für bestimmte Klassen vorherzusagen.

Hierzu wird ein \textit{S x S} Gitter über das Bild gelegt. Für jedes Feld in dem Gitter werden \textit{B} Bounding Boxen erzeugt. Jede Box besitzt neben den zum Gitterfeld relativen Positionswerten einen Wert für die Vorhersage der jeweiligen Klasse. Dieser Wert wird als \textit{confidence score} bezeichnet und wird durch die Multiplikation der Wahrscheinlichkeit für ein Objekt innerhalb der Box mit der \textit{Intersection over Union} (IoU) berechnet. Die IoU bildet die Präzision der berechneten Box im Verhältnis zu der Box aus den vortrainierten Testdaten. \cite[S. 2]{JosephRedmon.2016} 

Aus der Menge an Bounding Boxen werden schließlich mit Hilfe eines festgelegten Schwellwertes die Boxen mit gefundenen Objekten bestimmt (siehe Abbildung \ref{yolo_model}).

\begin{figure}[ht]
	\begin{center}
		\includegraphics[width=12cm]{Bilder/yolo_model.png} 
		\caption{Vereinfachte Darstellung der Objekterkennung mit dem YOLO Algorithmus \cite[S. 2]{JosephRedmon.2016}}
		\label{yolo_model}
	\end{center}
\end{figure}

\subsubsection{YOLOv2}

\subsubsection{YOLOv3}





\section{Datensatzformate} \label{format}

Basierend darauf, welcher Objektdetektor trainiert werden soll, muss der zum Training verwendete Datensatz in einem bestimmten Format vorliegen. Zum Trainieren des \textit{SSDs} wird das sogenannte \textit{Pascal Visual Object Classes} (PascalVOC) Format benötigt. 

Es definiert eine Unterteilung in \textit{JPEGImages}, \textit{Annotations} und \textit{ImageSets}. Während in dem Ordner \textit{JPEGImages} alle Bilder des Datensatzes vorhanden sind, befindet sich unter anderem die Information über die vorhandenen Objekte in dem Bild im Ordner \textit{Annotations}. Für jedes Bild des Datensatzes werden die Informationen in einer gleichnamigen XML-Datei abgelegt (siehe Listing \ref{pascalvoc:PascalVOC}).

\lstset{language=XML}
\lstinputlisting[
label=pascalvoc:PascalVOC,
caption=PascalVOC Bildannotation,
captionpos=b,
firstline=1,
lastline=26
]{Quellcode/annotation.xml}

Neben allgemeinen Metainformationen über das Bild befindet sich hier ebenso eine Liste aller markierten Objekte. Pro Objekt wird die Klassifikationskategorie, die Ausrichtung (z.B. \glqq Frontal\grqq{}), die Information über vollständiges Erscheinen im Bild, die Information über schwere Erkennbarkeit und die Bounding Box angegeben. Im Ordner \textit{ImageSets/Main} wird eine Unterteilung in Trainings-, Test- und Validierungsdatensatz durch drei Textdateien realisiert, die die Dateinamen der Bilddateien als Auflistung enthalten \cite{RenuKhandelwal.2019}. 

Das \textit{YOLO} Format für den \textit{YOLO} Objektdetektor definiert in einer \textit{.names}-Datei alle im Datensatz vorhandenen Kategorien durch simple Auflistung der Bezeichner. Die Bilder werden zusammen mit ihren Annotationen in einem separaten Ordner abgelegt. Die Annotationen folgen hier dem Format:

$<Kategorie-ID>\:<Zentrum-X>\:<Zentrum-Y>\:<Breite>\:<Hoehe>$

Die Unterteilung in Trainings- und Testdatensatz erfolgt durch Referenzierung der Bildpfade in zwei getrennten Textdateien. Schließlich wird in einer \textit{.data}-Datei der Pfad zu den beiden Textdateien und zur \textit{.names}-Datei sowie die Anzahl an Kategorien gespeichert \cite{ArunPonnusamy.20191006}.
\section{Cloud Infrastruktur}

Das Trainieren eines \textit{Deep Learning} Modells ist gerade bei großen CNN Architekturen äußerst rechenaufwendig. Tensor Operationen wie Matrixmultiplikationen und Konvolutionen erfordern im Rahmen des maschinellen Lernens hohe Parallelisierung und Taktfrequenzen, um in absehbarer Zeit gute Ergebnisse zu liefern. Die Rechenkapazität normaler Desktop-PCs reicht demnach meist nicht aus, um performantes \textit{Deep Learning} betreiben zu können. 

Abhilfe bieten Software-as-a-Service (SaaS) bzw. Platform-as-a-Service (PaaS) Angebote wie \textit{Amazon SageMaker}, \textit{Google Cloud Platform Cloud AI}, \textit{Azure ML Services} oder Start-ups wie \textit{FloydHub}. Diese bieten Infrastruktur in unterschiedlichen Zonen je nach Standpunkt der Rechenzentren zum Trainieren an sowie eine Plattform zum Verwalten der \textit{Deep Learning} Prozesse. 

\subsection{Trainingshardware}

Gerade GPUs bieten sich aufgrund ihres hohen Parallelisierungsvermögens gegenüber herkömmlichen CPUs an. Insbesondere \textit{NVIDIA} nimmt hierbei eine Vorreiterrolle in der Produktion von Server-GPUs ein. Die \textit{Compute Unified Device Architecture} (CUDA) von \textit{NVIDIA} ermöglicht hierbei als Programmiermodel und parallele Computing Plattform das Auslagern von Rechenprozessen auf GPUs. Das \textit{CUDA} Toolkit beinhaltet GPU beschleunigte Bibliotheken, einen Compiler, Entwicklungswerkzeuge sowie die eigentliche \textit{CUDA} Laufzeit und wird von vielen \textit{Deep Learning} Bibliotheken genutzt, wie z.B. \textit{PyTorch}. \cite{NVIDIA.20200209} \cite{PyTorch.20200209}

Vergleich man gängige GPUs, die oft in Rechenzentren der Hyperscaler angeboten werden, so ergibt sich folgende Tabelle \cite{TechPowerUp.20200209}:

\begin{tabular}[h]{l|c|c|c|c|c|c}
	& GTX 1080 & TITAN RTX & Tesla K80 & Tesla P100 & T4 & V100 \\
	\hline
	CUDA Cores & 2560 & 4608 & 2496 & 3584 & 2560 & 5120 \\
	Tensor Cores & / & 576 & / & / & 320 & 640 \\
	TFLOPS\footcite{Single Precision} & 8.873 & 16,31 & 4.113 & 9.526 & 8.141 & 14,13 \\
	Memory Bandwidth\footnote{in GB/sec} & 320,3 & 672 & 240,6 & 732,2 & 320 & 897 \\
	Max Power Consumption\footnote{During Normal Operation} & 180 & 280 & 300 & 250 & 70 & 300 \\
	\label{gpus}
\end{tabular}

Neben GPUs existieren seit 2015 die von Google entwickelten \textit{Tensor Processing Units} (TPUs). Diese Art von Spezialhardware erreicht pro TPU-Kern eine Rechenleistung von bis zu 92 TOPS \cite{HaraldBogeholz.20170406}. Schließt man 2048 solcher TPU-kerne zu einem TPU-Pod zusammen, so ergibt sich eine Rechenleistung von über 100 PetaFLOPS \cite{GoogleCloud.20200209}. 

\begin{figure}[ht]
	\begin{center}
		\includegraphics[width=14cm]{Bilder/tpu_comparison.png} 
		\caption[Vergleich V100 - TPU Pod]{Vergleich V100 - TPU Pod \cite{GoogleCloud.20200209b}}
		\label{tpu}
	\end{center}
\end{figure}

Eine weitere Steigerung versprechen Microsofts Field Programmable Gate Arrays (FPGAs), die allerdings nicht weiter im Rahmen dieser Arbeit betrachtet werden sollen \cite{KarlFreund.20170828}.

\subsection{FloydHub}

To be continued
\section{Drohnen} \label{drohnen}

Was die gesetzlichen Anforderungen zu Drohnen betrifft, so wurden im Juni 2019 von der \textit{European Aviation Safety Agency} (EASA) einheitliche Regeln veröffentlicht, die den Drohnenbetrieb in der Europäischen Union einheitlich regeln sollen. Da den Mitgliedsstaaten ein Zeitraum von einem Jahr zur Umsetzung der Regularien zugesprochen wurde, gelten in Deutschland weiterhin die Vorschriften der deutschen Drohnen-Verordnung von 2017 \cite{EASA.2019}.

Diese schreiben unter anderem vor \cite{Drohnen.de.2020}:
\begin{itemize}
	\item Eine Kennzeichnungspflicht ab einem Startgewicht von über 250g,
	\item Eine maximale Flughöhe von 100 Metern über dem Grund,
	\item Eine Haftpflichtversicherung und
	\item Flugverbotszonen (Wohngrundstücke, Naturschutzgebiete, etc.).
\end{itemize}

Analysiert man den Markt auf programmierbare Drohnen mit einem frei zugänglichen \textit{Software Development Kit} (SDK) und integrierter Kamera, so fällt das Angebot sehr gering aus. Im Folgenden soll ein Überblick über die zwei verfügbaren Modelle \textit{Ryze Tech Tello EDU} und \textit{Parrot Bebop 2} gegeben werden \cite{RyzeRobotics.2020, Parrot.20200520}.

\begin{center}
	\begin{tabular}[H]{l|c|c}
		& Ryze Tech Tello EDU & Parrot Bebop 2\\
		\hline
		Gewicht in Gramm & 87 & 500 \\
		Maximale Bildgröße & 2592x1936 & 4096x3072 \\
		Video Aufnahme Modi & 1280x720 30p & 1920×1080 30p \\
		Batterie Kapazität in mAh & 1100 & 2700 \\
		Max. Flugzeit in Minuten & 13 & 23 \\
		Max. Fluggeschwindigkeit in km/h & 28.8 & 60 \\
		Flugstabilisierung & Ja & Nein \\
		Preis in € & 159 & 295
	\end{tabular}
	\captionof{table}{Technische Daten zu programmierbaren Drohnen mit Kameraintegration}
	\label{table:drohnenspecs}
\end{center}

Der technische Vergleich (siehe Tabelle \ref{table:drohnenspecs}) bildet die Grundlage für die spätere Auswahl einer Drohne für die Umsetzung des \textit{SmartWarehouse} Szenarios.


\chapter{Konzeption}

Um eine Basis zur Umsetzung des \textit{Smart Warehouse} Szenarios zu schaffen, sind zunächst einige konzeptionelle Überlegungen notwendig, die in diesem Kapitel betrachtet werden sollen. Sie beschränken sich im Wesentlichen auf sechs übergeordnete Themenbereiche, die in Abbildung \ref{schritte} dargestellt sind. 

\begin{figure}[ht]
	\begin{center}
		\includegraphics[width=15cm]{Bilder/blockdiagramm.png} 
		\caption[Konzeptionelle Schritte]{Konzeptionelle Schritte}
		\label{schritte}
	\end{center}
\end{figure}

Diese sollen in folgenden Unterkapiteln nacheinander einzeln betrachtet werden. Daraus ergeben sich zwei Neuheitswerte. Zum einen soll evaluiert werden, ob die bestehenden Objektdetektoren für industrielle Anwendungsszenarien grundsätzlich geeignet sind, zum anderen ob das spezifische Anwendungsszenario zur automatisierten Durchführung einer Inventur von Warenhäusern mit einer Drohne prototypisch umsetzbar ist.

\section{Erstellen eines Traininsdatensatzes}

Das \textit{Smart Warehouse} lehnt sich an ein großes Warenhaus an, bei dem Produkte nicht in Kartons verpackt, sondern als ganzes auf Regalen angeordnet sind, ähnlich wie bei Warenhäusern wie \textit{Baumarkt} oder \textit{Selgros}. Bei dem Aufbau des Trainingsdatensatzes hat die Machbarkeitsstudie allerdings nicht zum Ziel, ein solches Warenhaus vollständig im Datensatz abzubilden, sondern wesentlich den Datensatz so umfangreich zu wählen, um eine generelle Umsetzbarkeit des \textit{Smart Warehouse} Szenarios zu beweisen. Im Rahmen des Projektes wurde sich deshalb exemplarisch auf Getränkeflaschen eines Warenhauses konzentriert. Dabei wurden neun Kategorien festgelegt (siehe Abbildung \ref{categories}). 

\begin{figure}[htb]
	\subfigure[Saskia Wasser Klein]{\includegraphics[angle=90,origin=c,width=0.30\textwidth]{Bilder/wasser_klein.jpg}}\hfill
	\subfigure[Saskia Wasser Groß]{\includegraphics[angle=90,origin=c,width=0.30\textwidth]{Bilder/wasser_gross.jpg}}\hfill
	\subfigure[Pepsi Cola Klein]{\includegraphics[angle=90,origin=c,width=0.30\textwidth]{Bilder/cola_klein.jpg}}\hfill
	\subfigure[Pepsi Cola Groß]{\includegraphics[angle=90,origin=c,width=0.30\textwidth]{Bilder/cola_gross.jpg}}\hfill
	\subfigure[ISO]{\includegraphics[angle=90,origin=c,width=0.30\textwidth]{Bilder/iso.jpg}}\hfill
	\subfigure[ACE]{\includegraphics[angle=90,origin=c,width=0.30\textwidth]{Bilder/ace.jpg}}\hfill
	\subfigure[Stenger Johannisbeerschorle]{\includegraphics[angle=90,origin=c,width=0.30\textwidth]{Bilder/johannisbeerschorle.jpg}}\hfill
	\subfigure[Stenger Apfelsaftschorle]{\includegraphics[angle=90,origin=c,width=0.30\textwidth]{Bilder/apfelsaftschorle.jpg}}\hfill
	\subfigure[Vitamalz Malzbier]{\includegraphics[angle=90,origin=c,width=0.30\textwidth]{Bilder/malzbier.jpg}}
	\caption{Die neun Datensatz-Kategorien}
	\label{categories}
\end{figure}

Der Datensatz besteht aus 1000 manuell annotierten Bildern. Die Bilder besitzen eine Auflösung von 2112x4608 Pixeln mit einer Farbtiefe von 24 Bit. Alle neuen Kategorien sind nahezu gleich häufig im Datensatz vorhanden. Durch das Tool \textit{LabelImg} wurden die Daten sowohl für das \textit{PascalVOC}, als auch das \textit{YOLO} Format annotiert. Im initialen Datensatz sind auf 75\% der Bilder die Objekte der jeweiligen Kategorien einzeln und klar erkennbar abgebildet. Hierdurch wird erhofft, dass Modell zunächst auf die Muster der jeweiligen Objekte zu trainieren. In 12,5\% der Bilder sind die Objekte der jeweiligen Kategorien ebenso einzeln, allerdings mit unterschiedlichen Hintergründen, Beleuchtungsverhältnissen, Blickwinkeln und Entfernungen abgebildet. Je nach Umgebung wurden Bilder dieses Anteils als schwer erkennbar markiert. Um das Warenhaus zu simulieren, sind in den letzten 12,5\% der Bilder die Objekte auf Regalen angeordnet, jeweils hintereinander oder in Getränkekästen (siehe Abbildung \ref{regal}). 

\begin{figure}[ht]
	\begin{center}
		\includegraphics[width=16cm]{Bilder/regal.jpg} 
		\caption[Smart Warehouse Regal]{SmartWarehouse Regal}
		\label{regal}
	\end{center}
\end{figure}
\chapter{Konzeption}

\section{Bewertungskriterien}

\begin{itemize}
	\item Kriterien für spätere Bewertung einführen
\end{itemize}

\section{Auswahl der Objektdetektoren} \label{detect}

Für das \textit{Smart Warehouse} Szenario soll eine Auswahl zwischen den vier Detektoren \textit{Faster R-CNN}, \textit{Mask R-CNN}, \textit{SSD} und \textit{YOLO} getroffen werden. Als Vergleichsbasis dienen die bereits veröffentlichten Benchmarkergebnisse. Zur Evaluation der Machbarkeitsstudie werden die zuvor eingeführten Bewertungskriterien auf die aus dieser Auswahl resultierenden Objektdetektoren angewendet.

\begin{figure}[ht]
	\begin{center}
		\includegraphics[width=12cm]{Bilder/ssd_results.png} 
		\caption[Vergleich SSD auf PascalVOC 2007]{Vergleich SSD auf PascalVOC 2007\footnotemark \cite{ssd.20161229}}
		\label{result}
	\end{center}
\end{figure}

Abbildung \ref{result} sind die Referenzergebnisse aus der wissenschaftlichen Veröffentlichung des \textit{SSDs} \cite{ssd.20161229}. Die Ergebnisse zeigen, wie die Objektdetektoren \textit{SSD}, \textit{YOLO}, \textit{Fast YOLO} und \textit{Faster R-CNN} untereinander abschneiden\footnotetext{Mask R-CNN wird in der wissenschaftlichen Veröffentlichung von SSD nicht aufgeführt. Beim Vergleich von Faster-RCNN (RoI-Align) mit Mask R-CNN ergibt sich eine mAP von 37.3 zu 38.2 auf Basis des COCO Datensatzes. Die FPS Anzahl betrug wesentlich 5 FPS \cite{KaimingHeGeorgiaGkioxariPiotrDollarRossGirshick.20180224}.}. Jeder der Detektoren besitzt eigene Charakteristika bezüglich der benötigten Auflösung der zu verarbeitenden Bilder, der Anzahl der generierten Bounding Boxen und der Batchgröße während des Trainings. Nach diesen Ergebnissen ist eindeutig festzustellen, dass der \textit{SSD} bezüglich \textit{mAP} mit 74.3\% bzw. 76.8\% am besten abschneidet. Unter Hinzunahme des \textit{PascalVOC} 2012 und \textit{Common Objects in Context} (COCO) \textit{trainval135k} Datensatzes erreicht der \textit{SSD} sogar das beste Ergebnis aus der ursprünglichen Veröffentlichung mit einer durchschnittlichen Präzision von 81.6\%. \textit{Faster-RCNN} kann zwar mit 73.2\% bezüglich der \textit{mAP} mithalten, ist allerdings mit nur 7 FPS nicht zur schnellen Inferenz ausgelegt. \textit{YOLO} schneidet in beiden Kategorien schlechter als der \textit{SSD} ab, er erzielt wesentlich eine \textit{mAP} von 66.4\% und eine Framerate von 21 FPS. Dem \textit{SSD} gelingt es also, ein gutes Verhältnis zwischen Präzision und Reaktionsvermögen zu bewahren. Durch den Verzicht auf den Schritt der Generierung von Bounding Box Vorschlägen und des Poolings kann \textit{SSD} deutlich schneller ablaufen als die Vergleichsdetektoren, während durch das Vordefinieren von Bounding Boxen ebenso eine hohe Präzision erzielt werden kann \cite{ssd.20161229}.

Allerdings ergibt sich vor allem für kleine Objekte ein erschwertes Detektionsvermögen, da diese in den höherliegenden Convolutional Layern untergehen. Als Lösung hierfür kann eine erhöhte Inputgröße gewählt werden (vgl. \textit{SSD512}) oder \textit{Data Augmentation} für den Lernprozess angewandt werden \cite{ssd.20161229}.

Diese Probleme treten bei Netzen der \textit{R-CNN} Familie nicht auf. Wird der \textit{Faster-RCNN} Objektdetektor mit \textit{Mask R-CNN} zur instanzbasierten Segmentierung auf Basis des COCO Datensatzes verglichen, so ergibt sich für \textit{Mask R-CNN} mit 38.2\% mAP nur eine geringe Verbesserung gegenüber \textit{Faster R-CNN} mit 37.3\% \cite{KaimingHeGeorgiaGkioxariPiotrDollarRossGirshick.20180224}. Für diesen Benchmark wurde wohlbemerkt das \textit{RoI-Pooling Layer} des \textit{Faster R-CNN} mit einem \textit{RoI-Align Layer} zur besseren Vergleichbarkeit mit dem \textit{Mask R-CNN} Objektdetektor ausgetauscht. Dennoch bleibt auch beim \textit{Mask R-CNN} das Problem eines langsameren Inferenzverhaltens gegenüber dem \textit{SSD} oder \textit{YOLO} offen. Für einen generell industriellen Einsatz könnte dieses langsame Inferenzverhalten möglicherweise problematisch sein, doch für das konkrete Szenario von stehenden Getränkeflaschen in der Machbarkeitsstudie ist eine starke Gewichtung der FPS Metrik zunächst mit Vorsicht zu betrachten \cite{IntanPurnamasar.20181215}. 

Ein weiteres Auswahlkriterium stellt dar, wie gut die Detektoren aufgesetzt und auf eigens erstellte Datensätze umkonfiguriert werden können. Nach Betrachtung mehrerer Repositories ließen sich der \textit{YOLO} und \textit{SSD} Objektdetektor einfach aufsetzen und auf eigene Datensätze anpassen, während bei \textit{Faster R-CNN} und \textit{Mask R-CNN} vermehrt auf Probleme gestoßen wurde. So ist Facebooks Implementierung von \textit{Mask R-CNN} \glqq Detectron\grqq{} beispielsweise nur auf Linux oder macOS lauffähig. Abgeleitete Repositories sind bereits als \textit{deprecated} deklariert und werden nicht mehr gewartet. Ein manuelles Aufsetzen dieser Implementierungen ist nur unter großem Aufwand möglich und wurde aufgrund der limitierten Zeit nicht weiter fortgeführt. Auch die Referenzimplementierung von \textit{Faster R-CNN} ist bereits als \textit{deprecated} deklariert und verweist auf die \textit{Mask R-CNN} Implementierung \textit{Detectron}. Nebenläufige Implementierungen sind ebenso als \textit{Legacy} Implementierungen vermerkt und nur zeitaufwändig manuell aufsetzbar, sofern sie die Windows-Plattform unterstützen.

Aufgrund dieser Umstände und des schlechteren Abschneidens in der zeitkritischen Modellinferenz wurden \textit{YOLO} und \textit{SSD} als die beiden Detektoren ausgewählt, um exemplarisch am \textit{Smart Warehouse Szenario} die Fähigkeit von Objektdetektoren zum industriellen Einsatz zu evaluieren. \textit{YOLO} wird im \textit{Darknet} Framework implementiert, dessen ursprüngliche Variante nicht mehr gewartet wird und zudem keine offizielle Dokumentation für die Verwendung unter Windows bereitsteht. \textit{YOLOv3} besitzt hingegen eine umfassende Dokumentation. \textit{Darknet} lässt sich einfach auf eigene Datensätze anpassen im Gegensatz zur Referenzimplementierung des \textit{SSD} im \textit{Caffe} Frameworks. Aus diesem Grund wurde sich bei \textit{SSD} für eine Custom Implementierung in \textit{PyTorch} entschieden.
\section{Auswahl der Trainingsinfrastruktur} \label{infrastructure}

Bei der Auswahl der Trainingsinfrastruktur wurden zunächst die Cloud PaaS-Angebote in Betrachtung gezogen. Diese ermöglichen meist eine weit bessere Performance als lokales Training. Wichtig bei der Auswahl war hierbei 

\begin{itemize}
	\item möglichst niedrige Betriebskosten,
	\item ein diverses Angebot an Hardware-Beschleunigern und
	\item ein einfaches Aufsetzen der Trainingsinfrastruktur.
\end{itemize}

Insbesondere sollten die Testversionen der jeweiligen Angebote zu Nutze gemacht werden, um niedrige Betriebskosten zu erreichen. Manche Testversionen bieten hierbei ein Startkontingent, das je nach ausgewählter Hardware unterschiedlich schnell bei Nutzung verbraucht wird, bei anderen Cloud Anbietern wird die Hardwarekonfiguration vorgegeben, die anschließend nur für eine bestimmte Zeitdauer unter Last genutzt werden kann. In Tabelle \ref{table:comparison} sind die Ergebnisse der Untersuchung dargestellt. 

\begin{center}
	\begin{tabular}[H]{l|c|c|c|c|c}
		& AWS & GCP & Azure & FloydHub & Colab \\
		\hline
		Nutzungsrahmen & 50 Std. & 300\$ & 200\$ & 2 Std. & Keine Beschränkung \\
		Hardware-Beschleuniger & Nein & Ja & Ja & Ja & Ja \\
		Setup-Komplexität & Hoch & Mittel & Mittel & Einfach & Einfach \\
	\end{tabular}
	\captionof{table}{Kostenlose SaaS-Angebote der Cloud Anbieter}
	\label{table:comparison}
\end{center}

\textit{Amazon SageMaker} bietet hierbei für 50 Stunden eine \textit{ml.m4.xlarge} Instanz für Modelltrainingszwecke an \cite{AmazonWebServices.2020}. Da diese allerdings nur 4 vCPUs und 16 GiB Arbeitsspeicher umfasst, also keinerlei Cloud GPU als Hardwarebeschleuniger angeboten wird, wurde das Angebot wieder verworfen \cite{AmazonWebServices.20200314b}.

Auf Empfehlung wurde anschließend die \textit{GCP} betrachtet. Diese bietet mit 300\$ Startguthaben für 12 Monate ein lukratives Angebot zum Ausprobieren von beliebigen \textit{GCP} Produkten \cite{GoogleCloudPlatform.20200314b}. Die Benutzung der \textit{Deep Learning VM} bietet zudem eine native Unterstützung des \textit{PyTorch} Frameworks, was von der \textit{SSD} Implementierung genutzt wird, und zugleich eine Auswahl aus vier gängigen Cloud GPUs, der \textit{NVIDIA Tesla K80}, \textit{NVIDIA Tesla P100}, \textit{NVIDIA Tesla T4} und der \textit{NVIDIA Tesla V100}. Um die Konfiguration der \textit{Deep Learning VM} allerdings mit Auswahl einer Cloud GPU abschließen zu können, muss zunächst das mit dem Account verknüpfte Kontingent erhöht werden. Hierzu musste an das \textit{GCP} Support Team ein offizieller Antrag gestellt werden. Aufgrund der geringen Kaufhistorie wurde der Antrag allerdings abgelehnt. 

\textit{Microsoft Azure} bietet für 200\$ bei einer Laufzeit von 30 Tagen Zugang zu allen \textit{Microsoft Azure} Diensten \cite{MicrosoftAzure.2020}. Darunter gehört eine \textit{NC6} Instanz mit sechs vCPUs und einer \textit{NVIDIA Tesla K80} \cite{MicrosoftAzure.202003124}. Da \textit{Microsoft Azures} Angebot allerdings nur sehr oberflächlich beschrieben wurde, wurde sich letzten Endes auch gegen \textit{Microsoft Azure} entschieden. 

Als letzter Anbieter wurde \textit{FloydHub} getestet. Hervorzuheben ist die besonders einfache Vorgehensweise bei der Account Erstellung und dem Aufsetzen der Trainingsinfrastruktur, was bereits in Kapitel \ref{cloud} beschrieben wurde. \textit{FloydHub} bietet 20 Stunden CPU Trainingszeit bzw. 2 Stunden GPU Trainingszeit auf einer \textit{NVIDIA Tesla K80} \cite{FloydHub.2020}. Neben einer \textit{NVIDIA Tesla K80} konnte ebenso Trainingszeit auf einer \textit{NVIDIA Tesla V100} erworben werden. Zudem wurde das verwendete \textit{PyTorch} Framework unterstützt. Aufgrund der einfachen Handhabung wurde sich trotz der erhöhten Kosten für \textit{FloydHub} entschieden. 

Während des Trainings mit der \textit{NVIDIA Tesla K80} fiel allerdings auf, dass die Wahl dieser GPU keine großen Performance Verbesserungen gegenüber lokalem Training brachte. Während lokal innerhalb einer Stunde 16 Epochen durchlaufen werden konnten, waren dies bei \textit{Floydhub} hingegen nur 8 Epochen. Dies veranlasste eine Gegenüberstellung gängiger Cloud GPUs mit lokalen GPUs, allen voran den bereits vorhandenen Desktop-Grafikkarten \textit{NVIDIA GeForce GTX 1080} und \textit{NVIDIA GeForce RTX 2060 SUPER} (siehe Tabelle \ref{table:hardware}) \cite{TechPowerUp.20200209}.

\begin{center}
	\begin{tabular}[H]{l|c|c|c|c|c|c}
		& K80 & P100 & T4 & V100 & GTX 1080 & RTX 2060 SUPER \\
		\hline
		CUDA Cores & 2496 & 3584 & 2560 & 5120 & 2560 & 2176 \\
		Tensor Cores & / & / & 320 & 640 & / & 272 \\
		TeraFLOPS (Single Precision) & 4,113 & 9,526 & 8,141 & 14,13 & 9,784 & 7,377 \\
		Memory Bandwidth (GB/sec) & 240,6 & 732,2 & 320 & 897 & 345,6 & 448 \\
		Suggested Power Supply Unit & 700 & 600 & 350 & 600 & 450 & 450
	\end{tabular}
	\captionof{table}{Vergleich von GPUs nach Rechenleistung}
	\label{table:hardware}
\end{center}

Hierbei fällt auf, dass im Grad der Parallelisierung eine \textit{NVIDIA Tesla K80} zwar mit den vorhandenen lokalen Grafikkarten mithalten kann, in der Anzahl an Rechenoperationen pro Sekunden allerdings weit schlechter abschneidet. Im Vergleich zu einer \textit{NVIDIA GeForce GTX 1080} schneidet die \textit{NVIDIA Tesla K80} bezüglich der Rechenleistung weniger als halb so schnell ab, was auch erklärt, warum nur halb so viele Epochen in einer Stunde auf der \textit{FloydHub} Cloud trainiert werden konnten. Damit sich das Training in der Cloud nach Performance lohnt, muss demnach mindestens eine \textit{NVIDIA Tesla V100} verwendet werden. Da diese allerdings mit 42\$ für zehn Stunden mehr als dreimal so teuer als eine \textit{NVIDIA Tesla K80} für 12\$ ist und zusätzlich zu den GPU Kosten noch monatliche Account-Gebühren berechnet werden\footnote{Je nach Account kann eine unterschiedliche Anzahl an Projekten erstellt und Speicherplatz verwendet werden. Die \textit{Beginner} Ausstattung von einem Projekt und 10 GB Speicher ist allerdings kostenfrei.}, wurde sich nach nun nach Kosten-Nutzen Abwägung letzten Endes auf lokales Training festgelegt. Dies ist ebenso hinsichtlich des Trainings des \textit{YOLO} Objektdetektors besser, da das sehr spezifische \textit{Darknet} Framework, das in der Implementierung genutzt wird, bisher von noch keinem Cloud Anbieter unterstützt wurde. Das Trainieren des \textit{YOLO} Objektdetektors in der Cloud hätte demnach eine Umentscheidung auf eine Alternativ-Implementierung in beispielsweise \textit{TensorFlow} oder \textit{PyTorch} nötig gemacht. Werden noch andere Anpassungen in der Programmlogik mit einbezogen, z.B. dass ein Zugriff auf das Dateisystem beim Erstellen von Dateien in der \textit{SSD} Implementierung in der Cloud Umgebung nicht möglich ist, so kommen zusätzlich zeitliche Bedenken mit auf. Ein lokales Trainieren bietet unter den genannten Voraussetzungen somit eine weitaus bessere Umgebung.

Auch wurden Überlegungen zum Training in \textit{Google Colab} unternommen, da hier ein einfaches Training mit TPUs ermöglicht werden kann. Da allerdings in \textit{Google Colab} nach 90 Minuten ein Trainingsjob beendet wird und die Rechenresourcen neuen Nutzern zugewiesen werden, war diese Art des Trainings ebenfalls nicht möglich. 

Aus arbeitstechnischen Gründen ist anzumerken, dass die beiden Objektdetektoren \textit{SSD} und \textit{YOLO} lokal jeweils auf unterschiedlicher Hardware trainiert werden, \textit{SSD} auf einer \textit{NVIDIA GeForce GTX 1080} und \textit{YOLO} auf einer \textit{NVIDIA GeForce RTX 2060 SUPER} GPU.

\section{Auswahl einer Drohne} \label{drone_selection}

Bei der Auswahl der Drohne sind vor allem zwei Bereiche zu betrachten:

\begin{itemize}
	\item Die gesetzlichen Rahmenbedingungen zur Drohne und
	\item Die technischen Anforderungen an die Drohne.
\end{itemize}

Aufgrund der gesetzlichen Auflagen aus Kapitel \ref{drohnen} wurde der Beschluss gefasst, dass die auszuwählende Drohne nur innerhalb von geschlossenen Wohnräumen im privaten Betrieb genutzt werden soll und zudem ein Startgewicht von unter 250g besitzen soll. Um eine studentische Machbarkeitsstudie durchführen zu können, ist eine solche eingeschränkte Anwendungsumgebung ausreichend. 

Die Programmierbarkeit der Drohne gehört zu der wichtigsten technischen Anforderung an die Drohne. Zudem soll sie eine integrierte Kamera aufweisen, die in der Lage ist, einen Videostream zur Laufzeit zur Verarbeitung bereit zu stellen. Hierzu wurden in Kapitel \ref{drohnen} bereits die beiden auf dem Markt verfügbaren Modelle  gegenübergestellt. Eine Wahl kann nur zwischen den beiden Modellen der \textit{Bebop 2} von der Firma Parrot oder der \textit{Tello EDU} Drohne von der Firma Ryze Tech getroffen werden. Die gesetzlichen Auflagen, speziell die Kennzeichnungspflicht, und die daraus abgeleiteten, projektkrelevanten Bedingungen lassen allerdings nur die \textit{Tello EDU} Drohne zu, zudem sie mit vertretbaren Budget erworben werden kann. Des Weiteren wird noch eine Garantie zu präzisem Schweben geboten, was gerade für die Objektdetektion vorteilhaft werden kann. Im Vergleich dazu erscheint die \textit{Parrot Bebop 2} vielmehr als eine Drohne, die für den Einsatz im Außenbereich konzipiert wurde.

---- Unteren Teil in 4.4 ------

Für die Steuerung sowie den Zugriff auf den Videostream bietet die \textit{Tello EDU} ein eigenes WiFi Netz an, zu dem sich das Steuergerät verbinden muss. Die Kommunikation erfolgt über das \textit{UDP} Protokoll\footnote{UDP steht für User Datagram Protocol. Es ist ein Transportprotokoll, welches im Gegensatz zu TCP verbindungslos und nicht zuverlässig ist.} und besteht aus mehreren Kommunikationskanälen: 

\begin{figure}[H]
	\begin{center}
		\includegraphics[width=8cm]{Bilder/communication_tello.pdf} 
		\caption{Kommunikationskanäle der Tello EDU Drohne}
		\label{communication_tello}
	\end{center}
\end{figure}

Der Video Stream und der Status der \textit{Tello EDU} werden unidirektional durch die Client Applikation bei der \textit{Tello EDU} abgefragt, wohingegen für das Senden von Befehlen eine bidirektionale Verbindung zum Einsatz kommt. Das liegt daran, dass die \textit{Tello EDU} auf jeden erhaltenen Befehl auch eine Antwort zurücksendet. Es existieren Befehle zur Steuerung und zum Auslesen von Informationen wie zum Beispiel der Geschwindigkeit oder Ladezustand der Batterie. Als Hilfestellung wird zusätzlich ein Beispielprogramm mitgeliefert, in welchem das Senden und Empfangen von Befehlen implementiert wurde \cite{RyzeTech.2018}. 
\section{Spezifikation der Inventursoftware}

Die Inventursoftware besteht aus einer einfachen Client-Server Anwendung. Die Client-Anwendung zeigt das Live-Drohnenbild mit den eingezeichneten, erkannten Objekten. Zudem wird die Anzahl an erkannten Objekten der spezifischen Klassen rechts daneben dargestellt. Wann die Drohne die Inventur durchführen soll, wird auf Initiative des Benutzers gestartet. 

\begin{figure}[ht]
	\begin{center}
		\includegraphics[width=15cm]{Bilder/UI.png} 
		\caption[SmartWarehouse User Interface]{SmartWarehouse User Interface}
		\label{ui}
	\end{center}
\end{figure}

Der Server, geschrieben im \textit{Django} Framework für Python, verbindet sich zu Drohne und weißt deren Flugsequenzen an. Auf ihm ist das \textit{Deep Learning} Modell deployt. Der Server inferiert die von der Drohne empfangenen Video-Stream-Frames mit dem Modell und gibt diese anschließend an die Client Applikation weiter. Eine \textit{REST} Schnittstelle gibt Aussage über die Bestandsdaten des Warenhauses.

\chapter{Realisierung}

\section{Trainieren der Objektdetektoren}

\begin{itemize}
	\item Bilder mit Labeln in der Arbeit
	\item Soll beweisen, dass etwas entstanden ist (Krassen Eindruck vermitteln)
	\item Keine Ergebnisse evaluieren, keine Detektoren bewerten
\end{itemize}

\section{Drohnen Anbindung}

\section{Dashboard Entwicklung}


\chapter{Ergebnisse} \label{evaluation}

In diesem Kapitel werden die Ergebnisse des Trainings und des Einsatzes der Objektdetektoren nach den in Kapitel \ref{eval} definierten Kriterien dargestellt. Es beinhaltet die Ergebnisse zur Präzision, zum Inferenzverhalten, zum Reaktionsvermögen und zum Trainingsverhalten der Detektoren.

\section{Präzision und Inferenzverhalten}

\subsection*{SSD}

Ursprünglich wurden 500 Epochen für das Training vorgesehen. Da allerdings beim Training schon nach knapp über hundert Epochen sich der Gradient der Kostenfunktion nur träge veränderte, wurde im Sinne des \textit{Early Stoppings} nach 121 Epochen das Training vorzeitig beendet, um \textit{Overfitting} zu vermeiden. Abbildung \ref{ssdloss} zeigt den Verlauf der Trainingsverlustkurve und der Testverlustkurve während des Kreuzvalidierungsverfahrens.

\begin{figure}[H]
	\begin{center}
		\includegraphics[width=8cm]{Bilder/ssdloss.jpeg} 
		\caption{Entwicklung der SSD Trainings- und Testverlustkurve während dem Training}
		\label{ssdloss}
	\end{center}
\end{figure}

Die Entscheidung zum \textit{Early Stopping} basiert auf dem Anstieg der Differenz zwischen Trainings- und Testverlustkurve ab Epoche 122, was auf \textit{Overfitting} schließen lässt. Zur Epoche davor war die Differenz der beiden Kurven am niedrigsten bei vergleichsweise geringem Verlust im Trainingsverfahren. Auffällig ist ebenfalls der große Abfall der Trainingsverlustkurve bei Epoche 45 und Epoche 89. Hier findet ein Wechsel im Kreuzvalidierungsverfahren statt. Die bessere Generalisierungsfähigkeit des Modells bei neuen Testdaten zeigt Ausschlag, indem die Klassifikationsergebnisse schlagartig besser werden und zu niedrigeren Kosten führen. Auch in Epoche 67 ist einer dieser Ausschläge zu sehen, der allerdings kleiner im Vergleich zu den anderen ausfällt. 

Zur Epoche 121 betrug das Ergebnis der Kostenfunktion 1.7. Es ergab eine \textit{mAP} von 83.1\%, leicht über den Referenzergebnissen von \textit{SSD} zu \textit{PascalVOC} (siehe Abbildung \ref{result}). Die Ergebnisse zu den einzelnen Klassen sind in folgender Tabelle dargestellt:

\begin{center}
	\begin{tabular}[H]{l|c}
		Klasse & mAP \\
		\hline
		Saskia Wasser Groß & 77.62\% \\
		Saskia Wasser Klein & 75.96\% \\
		Pepsi Cola Groß & 94.94\% \\
		Pepsi Cola Klein & 86.38\% \\
		ISO & 86.37\% \\
		ACE & 85.43\% \\
		Stenger Johannisbeerschorle & 69.47\% \\
		Stenger Apfelsaftschorle & 82.48 \% \\
		Vitamalz Malzbier & 76.24\%
	\end{tabular}
	\captionof{table}{Validierungsergebnisse SSD}
	\label{table:ssdresults}
\end{center}

Wird nun das trainierte Modell auf echte Daten angewendet, so fällt auf, dass manche Objekte doppelt detektiert werden. Um dieses Problem zu lösen, gibt es zwei Möglichkeiten. 

Als erstes kann bei der Detektion der minimale \textit{confidence score} angegeben werden, ab wann eine Detektion offiziell als solche wahrgenommen wird. Hier liegt die Herausforderung darin, einen optimalen Wert zu finden, sodass verdeckte Objekte noch als solche erkannt werden, aber doppelt erkannte Objekte nicht mehr auftreten. Der \textit{confidence score} wurde nach mehrmaligem Iterieren auf 0.7 festgelegt.

Die zweite Möglichkeit besteht darin, die maximale Überlappung zweier Bounding Boxen festzulegen. Somit werden doppelte Bounding Boxen, die sich flächenmäßig über einem gewissen Grenzwert überlappen, auf eine Bounding Box reduziert. Er stellte ich sich Parameter von 0.45 als geeignet heraus. Außerdem wurden Inkonsistenzen im Detektionsverhalten festgestellt.

\begin{figure}[H]
	\subfigure[Verdecktes Objekt I]{\includegraphics[width=0.20\textwidth]{Bilder/verdeckt.jpeg}}\hfill
	\subfigure[Verdecktes Objekt II]{\includegraphics[width=0.20\textwidth]{Bilder/verdeckt2.jpeg}}\hfill
	\subfigure[Extreme Blicklagen I]{\includegraphics[width=0.20\textwidth]{Bilder/winkel.jpeg}}\hfill
	\subfigure[Extremen Blicklagen II]{\includegraphics[width=0.20\textwidth]{Bilder/winkel2.jpeg}}
	\caption{Detektionsverhalten von SSD bei extremen Blicklagen}
	\label{lagen}
\end{figure}

So sind von anderen Objekten verdeckte Objekte nur schwer zu erkennen. So wird beispielsweise in Abbildung \ref{lagen} (a) das rechte hintere Objekt selbst bei einem sehr niedrigen \textit{confidence score} Schwellwert nicht erkannt. In Abbildung \ref{lagen} (b) hingegen ist eine Detektion ab einem \textit{confidence score} von 0.53 möglich, was allerdings wieder weitere ungenaue und ungewünschte Detektionen verursacht. Ähnliche Probleme ergeben sich für die Detektion von Objekte aus extremen Blicklagen. In Abbildung \ref{lagen} (c) werden die Objekte beispielsweise erst ab einem \textit{confidence score} von 0.54 erkannt, allerdings nicht als drei einzelne Objekte, sondern als ein großes Gesamtobjekt. Dies gilt allerdings nicht für alle Klassen. Bei Anwendung dieses Problemfalls auf eine andere Klasse ergeben sich bei vergleichbarem \textit{confidence score} von 0.56 sehenswertere Ergebnisse (siehe Abbildung \ref{lagen} (d)). 

Die zwei genannten Problemfälle wurden im Datensatz zwar zu 12.5\% abgedeckt, scheinen allerdings nur wenig Auswirkung auf besseres Detektionsverhalten für solche Fälle geliefert zu haben. Es lässt sich auch keine Aussage darüber treffen, ob eine Erweiterung des Datensatzes mit weiteren solchen Extremfällen eine Abhilfe für dieses Problem hätte liefern können. 

\begin{figure}[H]
	\subfigure[1 Meter]{\includegraphics[width=0.30\textwidth]{Bilder/einmeter.jpeg}}\hfill
	\subfigure[2 Meter]{\includegraphics[width=0.30\textwidth]{Bilder/zweimeter.jpeg}}\hfill
	\subfigure[3 Meter]{\includegraphics[width=0.30\textwidth]{Bilder/dreimeter.jpeg}}\hfill
	\caption{Detektionsverhalten von SSD bei unterschiedlichen Entfernungen}
	\label{entfernung}
\end{figure}

Auch die Entfernung zum zu detektierenden Objekt besitzt eine Auswirkung auf das Detektionsverhalten. In Abbildung \ref{entfernung} wird gezeigt, dass ab einer Entfernung von drei Metern eine zuverlässige Detektion nicht mehr möglich ist. In diesem Beispiel ist erst ab einer niedrigeren \textit{confidence score} Grenze von 0.31 eine Detektion des Objekts wieder erfolgt, bei anderen Beispielen sogar erst ab einem Wert von 0.09.

\begin{figure}[H]
	\subfigure[überbeleuchtet I]{\includegraphics[width=0.20\textwidth]{Bilder/ueberbeleuchtet2.jpeg}}\hfill
	\subfigure[überbeleuchtet II]{\includegraphics[width=0.20\textwidth]{Bilder/ueberbeleuchtet.jpeg}}\hfill
	\subfigure[normal]{\includegraphics[width=0.20\textwidth]{Bilder/normalbeleuchtet.jpeg}}\hfill
	\subfigure[unterbeleuchtet]{\includegraphics[width=0.20\textwidth]{Bilder/unterbeleuchtet.jpeg}}\hfill
	\caption{Detektionsverhalten von SSD bei unterschiedlichen Beleuchtungsverhältnissen}
	\label{sicht}
\end{figure}

Des Weiteren haben unterschiedliche Beleuchtungsgrade eine Auswirkung auf das Detektionsverhalten. In Abbildung \ref{sicht} wird unterschieden zwischen dem Detektionsverhalten bei überbeleuchteten, normalen und unterbeleuchteten Sichtverhältnissen. Überbeleuchtete Umgebungsverhältnisse erschweren die Objektdetektion. Zum einen können bestimmte Objekte gar nicht mehr erkannt werden (siehe Abbildung \ref{sicht} (b)) oder die dadurch verursachte Lichtabsorption und anschließende Emission durch die zu detektierenden Objekte führt zu einer Falschdetektion. In Abbildung \ref{sicht} (a) wird so beispielsweise mit einem \textit{confidence score} von 0.91 das Objekt der Klasse \textit{ACE} falsch als \textit{ISO} klassifiziert.

Alle Detektionen reagierten allerdings invariant gegenüber unterschiedlichen Hintergründen oder Bildauflösungen.

\subsection*{YOLO}

Um eine möglichst gute Aussage über den Vergleich der beiden Objektdetektoren treffen zu können, kommt bei \textit{YOLO} der exakt gleiche Trainings- und Testdatensatz wie beim \textit{SSD} zum Einsatz. Das Training von \textit{YOLO} ergab eine \textit{mAP} von 80.36\%. Abbildung \ref{yolo_result} zeigt die Verbesserung des Modells während dem Trainingsprozess. 

\begin{figure}[H]
	\begin{center}
		\includegraphics[width=8cm]{Bilder/yolo_result.png} 
		\caption{Entwicklung der YOLO Testverlustkurve und der \textit{mAP} während dem Training}
		\label{yolo_result}
	\end{center}
\end{figure}

Wie beim \textit{SSD} erreicht das Modell nach bereits etwa 4500 Batches eine \textit{mAP} von 80\%. Da sich das Modell ab diesem Zeitpunkt nicht mehr maßgeblich verbessert hat, wurde das Training vor Erreichung der durch die Dokumentation nahegelegte maximale Anzahl an Batches abgebrochen. Das frühe Erreichen eines sehr guten Ergebnis kann darauf zurückgeführt werden, dass es sich bei den gelabelten Klassen um Objekte des gleichen Typs handelt und das Modell dementsprechend leichter trainiert werden kann. 

Der \textit{confidence score} liegt standardmäßig bei 0.25. Nach mehreren Durchläufen stellt sich heraus, dass dieser Wert nicht verändert werden sollte. Ein zu geringer Wert sorgt zwar dafür, dass möglicherweise mehr Objekte erkannt werden, allerdings steigt damit auch die Rate an falsch oder doppelt erkannter Objekte. Ein Parameter für die Obergrenze der Überlappung zweier Bounding Boxen existiert nicht.

Tabelle \ref{table:yoloresults} zeigt die \textit{mAP} für jede Klasse.

\begin{center}
	\begin{tabular}[H]{l|c|c}
		Klasse & mAP & Differenz zu SSD\\
		\hline
		Saskia Wasser Groß & 76.69\% & (-0.93\%) \\
		Saskia Wasser Klein & 80.63\% & (+4.67\%) \\
		Pepsi Cola Groß & 73.14\% & (-21.8\%) \\
		Pepsi Cola Klein & 75.24\% & (-11.14\%) \\
		ISO & 90.28\% & (+3.91\%) \\
		ACE & 86.69\% & (+1.26\%) \\
		Stenger Johannisbeerschorle & 72.50\% & (+3.03\%) \\
		Stenger Apfelsaftschorle & 78.05\% & (-4.43\%) \\
		Vitamalz Malzbier & 81.94\% & (+5.7\%)
	\end{tabular}
	\captionof{table}{Validierungsergebnisse YOLO im Vergleich zu SSD}
	\label{table:yoloresults}
\end{center}

Auch bei \textit{YOLO} können verdeckte Objekte nur schwer erkannt werden. Ein Unterschied zu \textit{SSD} zeigt sich bei der Ansicht von oben:

\begin{figure}[H]
	\centering
	\subfigure[verdeckt]{\includegraphics[width=0.30\textwidth]{Bilder/yolo_verdeckt.jpg}}
	\hspace{2cm}
	\subfigure[winkel]{\includegraphics[width=0.30\textwidth]{Bilder/yolo_winkel.jpg}}
	\caption{Detektionsverhalten von YOLO bei extremen Blicklagen}
	\label{lagen_yolo}
\end{figure}

Hier werden zwei von drei Flaschen erkannt, die allerdings der falschen Klasse (\textit{Saskia Wasser Groß} statt \textit{Saskia Wasser Klein}) zugeordnet werden und eine der zwei Bounding Boxes zwei Flaschen umschließt.

\begin{figure}[H]
	\subfigure[1 Meter]{\includegraphics[width=0.30\textwidth]{Bilder/yolo_entfernung1.jpg}}\hfill
	\subfigure[2 Meter]{\includegraphics[width=0.30\textwidth]{Bilder/yolo_entfernung2.jpg}}\hfill
	\subfigure[3 Meter]{\includegraphics[width=0.30\textwidth]{Bilder/yolo_entfernung3.jpg}}\hfill
	\caption{Detektionsverhalten von YOLO bei unterschiedlichen Entfernungen}
	\label{entfernung_yolo}
\end{figure}

Wie bei \textit{SSD} erfolgt nach bereits drei Metern Entfernung keine Detektion mehr (siehe Abbildung \ref{entfernung_yolo}). Der \textit{confidence score} für die korrekt erkannte Bounding Box liegt bei nur 2\%.
 
\begin{figure}[H]
 	\subfigure[überbeleuchtet]{\includegraphics[width=0.30\textwidth]{Bilder/yolo_beleuchtung2.jpg}}\hfill
 	\subfigure[normal]{\includegraphics[width=0.30\textwidth]{Bilder/yolo_beleuchtung1.jpg}}\hfill
 	\subfigure[unterbeleuchtet]{\includegraphics[width=0.30\textwidth]{Bilder/yolo_beleuchtung3.jpg}}\hfill
 	\caption{Detektionsverhalten von YOLO bei unterschiedlichen Beleuchtungsverhältnissen}
 	\label{sicht_yolo}
\end{figure}

Beim Testen der verschiedenen Beleuchtungsverhältnisse fällt auf, dass im Gegensatz zum \textit{SSD} die Flasche in der unterbelichteten Umgebung nicht erkannt wird beziehungsweise mit einem \textit{confidence score} von nur 6\%. 

Gleich zu \textit{SSD} reagiert \textit{YOLO} auch invariant gegenüber unterschiedlichen Hintergründen oder Bildauflösungen.
 
\section{Reaktionsvermögen}

Bei der Inferenz fällt allerdings entgegen der Erwartungen auf, dass die Inferenz überdurchschnittlich langsam verläuft. Das Problem lässt sich auf die synchrone Arbeitsweise der bisherigen Detektionsalgorithmen zurück führen, bei dem erst ein neuer Frame des Videostreams angefragt wird, sobald das aktuelle Bild durch die Vorverarbeitung gelaufen ist und durch das Modell inferiert wurde. 

Um dem entgegen zu wirken, wurde ein Bufferkonzept in einem parallelem Thread realisiert, der einzelne Frames zeitgleich zur Inferenz anfragt und zwischenspeichert. Ist der Buffer voll, so werden nach dem \textit{First In First Out} Verfahren die älteren Frames verworfen. Dadurch konnte die FPS Anzahl von \textit{SSD} von maximal 18 auf die vollen 30 und bei \textit{YOLO} von 20 auf 28 gesteigert werden. Dadurch sollten sämtliche Änderungen in der Umgebung rechtzeitig vom Objektdetektor erkannt werden. 

Ein weiteres Problem beschreibt die initiale Latenz zwischen der Inferenz und der Bildaufnahme. Die Inferenz kann erst gestartet werden, sobald die Gewichtungen des Modells initialisiert und geladen wurden. Bei \textit{YOLO} benötigt die Initialisierung etwa drei Sekunden, bei \textit{SSD} vier Sekunden. Das lässt sich umgehen, indem entweder der Thread zur Bildaufnahme verzögern gestartet wird oder dessen Buffer kleiner gewählt wird.

\section{Trainingsverhalten}

\begin{center}
	\begin{tabular}[H]{l|c|c|c|c}
		Detektor & Hardware & Batch Größe & Dauer Epoche & Dauer Insgesamt \\
		\hline
		SSD & GTX 1080 & 16 & 9 Min. & 16.5 Std. \\
		YOLO & RTX 2060 & 64 & 1.5 Min. & 8 Std.
	\end{tabular}
	\captionof{table}{Trainingsverhalten von SSD und YOLO}
	\label{table:duration}
\end{center}

Tabelle \ref{table:duration} zeigt die Ergebnisse des Trainingsverhaltens von \textit{SSD} und \textit{YOLO}. Pro Epoche wurden 998 Bilder durchlaufen. Die Dauer der beiden Trainingsdurchläufe lässt sich allerdings nur schwer vergleichen. Zum einen wird für das Training von \textit{YOLO} eine andere GPU verwendet, deren Tensor Kerne den Trainingsprozess merklich beschleunigen. Zum anderen werden die beiden Objektdetektoren in zwei verschiedenen Frameworks implementiert, wodurch ein anderes Trainingsverhalten von Grund auf gegeben ist und Unterschiede in der Performance auftreten können. Auch die Batch Größe wurde unterschiedlich gewählt, was sich auf die Häufigkeit des Gradientenabstiegs auswirkt. Bei beiden Verfahren kann das Training durch die Ablage von sogenannten Modell-Checkpoints zu einem späteren Zeitpunkt fortgeführt werden. Das ist vor allem dann von Vorteil, wenn nachträglich Parameter oder Trainingsdaten angepasst werden und kein kompletter Neustart des Trainings erforderlich ist.


\chapter{Diskussion}

\section{Einsatz der Objektdetektoren in der Industrie}

Die Bewertung der beiden Detektoren für den industriellen Einsatz baut auf den in Kapitel \ref{eval} eingeführten Bewertungskriterien auf. 

Hinsichtlich der Präzision fallen beide Objektdetektoren besser als die ursprünglichen Referenzergebnisse aus den wissenschaftlichen Veröffentlichungen aus. Es ist aber zu bemerken, dass die Ergebnisse nicht gut mit denen der wissenschaftlichen Veröffentlichungen vergleichbar sind, da sie auf unterschiedlichen, deutlich einfacheren Daten trainiert wurden. Der Vergleichsdatensatz ist hierbei \textit{PascalVOC 2007}, der 20 Klassen besitzt und damit für das Modell komplexer zu modellieren ist, als der \textit{Smart Warehouse} Datensatz mit nur 9 Klassen. Mit 83,1\% fällt \textit{SSD300} um 8,8\% besser aus, während \textit{YOLOv3} sogar eine Verbesserung von 16.7\% verzeichnet. Dennoch lässt sich darauf schließen, dass beide Implementierungen bezüglich Präzision in einer Größenordnung skalieren, die dem Niveau der Vergleichsergebnisse aus \textit{PascalVOC 2007} nachkommt. Die leicht erhöhte Präzision von \textit{SSD300} gegenüber \textit{YOLOv3} liegt daran, dass \textit{SSD300} im Gegensatz zu \textit{YOLOv3} Bounding Box Vorschläge unterschiedlicher Seitenverhältnisse zulässt und die Unterteilung in Gitterstrukturen ebenfalls für mehrere Skalierungen durchgeführt wird. Dies wird vor allem erkenntlich beim Vergleich der \textit{mAP} bei den Klassen \textit{Pepsi Cola Groß} und \textit{Pepsi Cola Klein}. Objekte der Klassen sehen grundsätzlich gleich aus, nur die Größe ändert sich, sind also unterschiedlich skaliert. Lässt man den Effekt bei unterschiedlichen Skalierungen allerdings außer Auge, so schneidet \textit{YOLOv3} dennoch besser ab als \textit{SSD300} auf Basis des \textit{Smart Warehouse} Datensatzes (siehe Tabelle \ref{table:yoloresults}).

Ein weiteres Bewertungskriterium ist das Reaktionsvermögen. Hier schneidet \textit{SSD300} mit durchschnittlich 30 FPS leicht besser ab als \textit{YOLOv3} mit 28 FPS, wobei dieser Unterschied kaum als wahres Entscheidungskriterium gesehen werden sollte, \textit{SSD300} \textit{YOLOv3} vorzuziehen. Auf einem leichten Datensatz wie \textit{Smart Warehouse} gibt es also kaum Unterschiede im Reaktionsvermögen als im Vergleich zu größeren und komplexeren Datensätzen wie \textit{PascalVOC 2007} (siehe Abbildung \ref{result}). Laut wissenschaftlichen Referenzergebnissen scheidet hier \textit{SSD300} um 25 FPS besser ab als \textit{YOLO}.

Bezüglich Trainingsverhalten könnte \textit{SSD300} das Training mit leistungsfähigeren Grafikkarten wie der \textit{NVIDIA Tesla V100} von 16,5 Stunden auf umgerechnet 5,2 Stunden geschätzt rund drei Mal schneller vollziehen. Die Rechnung legt die Tabelle \ref{table:comparison} zugrunde. Pro Sekunde kann eine \textit{NVIDIA GeForce GTX 1080} hierbei $2560\cdot 9,784 = 25.047,04 TeraFLOPs$ vollziehen im Vergleich zu $5120\cdot 14,13 = 72.345,6 TeraFLOPs$ pro Sekunde bei einer \textit{NVIDIA Tesla V100}. Da die \textit{NVIDIA Tesla V100} zusätzlich noch über Tensor Cores verfügt, kann der Zeitgewinn sogar noch höher eingeschätzt werden. Bei acht \textit{NVIDIA Tesla V100} betragen die Trainingskosten so beispielsweise nur noch ca. 39 Minuten. Somit lässt sich festhalten, dass sich \textit{SSD300} durchaus mit aktuellen Cloud Grafikkarten effizient trainieren lassen lässt. Da die Custom-Implementierung allerdings Hilfsstrukturen auf dem Sekundärspeicher erstellt, können hierbei beim Trainieren auf der Cloud Probleme auftreten. Meistens wird der Sekundärspeicher auf virtuellen Maschinen in der Cloud zum Schutz vor Angriffen nur \textit{read-only} bereitgestellt. 

\textit{YOLOv3} ließ sich mit acht Stunden deutlich schneller trainieren, allerdings unter vollkommen anderen Voraussetzungen. Nicht nur die verwendeten \textit{Deep Learning} Frameworks sind unterschiedlich, sondern wie bereits erläutert aus Gründen der Arbeitsteilung ebenso die Trainingshardware. Als Grafikkarte wurde bei \textit{YOLOv3} eine \textit{NVIDIA GeForce RTX 2060} verwendet, die zusätzlich über 272 Tensor Cores verfügt und somit die Rechenleistung erhöht. Auch musste bei \textit{YOLOv3} nicht die Batchgröße erniedrigt werden wie bei \textit{SSD300}, um einem GPU Speicherüberlauf entgegen zu wirken. \textit{YOLO} löst dieses Problem allgemein durch das \textit{Subdivision} Konzept, welches bei \textit{SSD} nicht existiert. Bei \textit{SSD300} mussten somit durch die geringere Batchgröße zwangsmäßig mehr Gradientenabstiege vollzogen werden, was zwar auch auf die größere Stabilität hinweist, aber sich als Seiteneffekt auch auf die Trainingslänge auswirkt. Ein Training in der Cloud ist allerdings mit hoher Wahrscheinlichkeit immer noch schneller, z.B. auf Basis einer \textit{NVIDIA Tesla V100}, deren technische Daten zur Rechenleistung rund doppelt so gut sind wie die zur \textit{NVIDIA GeForce RTX 2060} (siehe Tabelle \ref{table:comparison}). 

Zuletzt muss das Inferenzverhalten betrachtet werden. Hierzu kann folgende Tabelle \ref{table:inference} herangezogen werden:

\begin{center}
	\begin{tabular}[h]{l|c|c}
		Kriterium & SSD300 & YOLOv3 \\
		\hline
		Beleuchtung & 0 & 0 \\
		Extreme Blicklagen & 0 & 0 \\
		Entfernung & 0 & 0 \\
		Verdeckung & - & - \\
		Doppelte Erkennung & + & + \\
		Hintergrund & + & + \\
		Bildauflösung & + & + \\
	\end{tabular}
	\captionof{table}{Inferenzverhalten SSD300 und YOLOv3}
	\label{table:inference}
\end{center}

Hinsichtlich der Beleuchtung lässt sich nur eine Hypothese aufstellen, dass gerade sehr transparente Objekte durch die starke Überbeleuchtung nur schwer durch Mustererkennung detektierbar sind. Sowohl bei \textit{SSD} als auch bei \textit{YOLO} konnte so beispielsweise die Klasse \textit{Saksia Wasser Klein} nicht detektiert werden. Welche Features letztendlich das neuronale Netz zur Detektion heranzieht, lässt sich nicht sagen und damit die Hypothese nicht beweisen. Während \textit{SSD} im Falle der Überbeleuchtung zudem das Problem von Falschdetektionen aufweist, werden bei \textit{YOLO} Objekte erst ab einer niedrigeren \textit{confidence score} Grenze erkannt. Beim Test auf Unterbeleuchtung konnte \textit{SSD300} hingegen Objekte besser erkennen als \textit{YOLOv3}. Das Verhältnis im Detektionsvermögen bei variierenden Beleuchtungsverhältnissen ist somit unter beiden eher ausgeglichen. Die Ergebnisse stellen allerdings auch die Frage auf, ob bei Erweiterung des Datensatzes um weitere Bilder mit unterschiedlichen Beleuchtungsverhältnissen ein besseres Detektionsvermögen ermöglicht werden könnte.

Gleiches gilt bei extremen Blicklagen. Hier entsteht bei \textit{YOLO} das Problem von Falschdetektionen, bei beiden Detektoren, dass mehrere Objekte als ein Objekt erkannt werden. Abbildung \ref{lagen} (d) ist hingegen ein gutes Beispiel für die Machbarkeit der Detektion in extremen Blicklagen. Auch hier lässt sich vermuten, dass eine Erweiterung des Datensatzes Abhilfe leisten könnte.

Auch scheint die Detektion ab gewissen Entfernungen und Verdeckungsgraden nur erschwert möglich zu sein, bei \textit{SSD} ab drei Metern, bei \textit{YOLO} hingegen schon ab zwei Metern.

Eine Ausweitung des Trainingsdatensatzes erscheint demnach nicht abwegig. Im \textit{Smart Warehouse} Datensatz sind die behandelten Fälle mit 12.5\% nur unterbesetzt abgebildet. Neben einer Erweiterung des Datensatzes wäre ebenso das Anwenden von \textit{Data Augmentation} oder \textit{Transfer Learning} eine Alternative, mit der der Generalisierungsgrad des Modells gesteigert werden könnte. Das Problem der Detektion von verdeckten Objekten scheint aber bei beiden Objektdetektoren vorzuliegen. Auch hier könnte die Erweiterung des Datensatzes dazu dienen, dass die Modelle neue Merkmale gerade von teilweise verdeckten Objekten erlernen könnten. Ob das Erkennen von Teilmerkmalen allerdings ausreicht, um einen genügend hohen \textit{confidence score} zu entwickeln, bleibt offen. Das Problem doppelter Erkennung, wird von beiden allerdings gut gelöst, durch das Festlegen eines Schwellwertes im \textit{confidence score}. Auch scheinen keine Merkmale aus Hintergrundinformationen zur Detektion heran gezogen worden sein, da das Detektionsverhalten von \textit{SSD300} als auch von \textit{YOLOv3} invariant zu unterschiedlichen Hintergründen war. Auch die Bildauflösung hat keine Auswirkung auf das Detektionsvermögen.

\begin{center}
	\begin{tabular}[h]{l|c|c}
		Kriterium & SSD300 & YOLOv3 \\
		\hline
		Präzision & + & + \\
		Reaktionsvermögen & + & + \\
		Trainingsverhalten & 0 & + \\
		Inferenzverhalten & 0 & 0 \\
	\end{tabular}
	\captionof{table}{Gesamtbewertung SSD300 und YOLOv3}
	\label{table:discussion}
\end{center}

Insgesamt lässt sich schließen, dass sowohl \textit{SSD300} als auch \textit{YOLOv3} Objektdetektoren für den industriellen Einsatz darstellen. Die zuvor besprochenen Bewertungskriterien sind in Tabelle \ref{table:discussion} abschließend zusammengefasst. Welcher der beiden Detektoren präferiert genutzt werden soll, lässt sich generell nicht aussagen. Beide Detektoren haben ihre Stärken in unterschiedlichen Anwendungsgebieten. 

Falls viele Objekte unterschiedlicher Skalierung in der Detektionsumgebung erkannt werden sollen, so ist \textit{SSD300} \textit{YOLOv3} vorzuziehen. Falls dies allerdings nicht der Fall ist und bei einfachen Datensätzen wie \textit{Smart Warehouse} mehr Wert auf Genauigkeit gelegt wird, so ist \textit{YOLOv3} die bessere Wahl. Im Reaktionsvermögen sind beide bei einfachen Datensätzen circa gleich schnell, ein Unterschied lässt sich anscheinend erst bei komplexeren Datensätzen erkennen. Bezüglich des Trainingsverhaltens ist \textit{YOLOv3} \textit{SSD300} vorzuziehen, nicht nur aufgrund der Möglichkeit \textit{Subdivisions} zu definieren, sondern auch aufgrund der einfacheren Aufsetzbarkeit. Während \textit{YOLO} die Menge an in den Grafikspeicher zu ladenden Bildern eines Batches durch die \textit{Subdivisions} partitionieren kann, um einen Speicherüberlauf zu vermeiden, muss bei \textit{SSD} hingegen immer die gesamte Batchgröße an Bildern in den GPU Speicher geladen werden. Andernfalls kann bei \textit{SSD} nur die Batchgröße reduziert werden, was hinsichtlich des Gradientenverfahren womöglich nicht gewünscht sein kann. Außerdem ist es nicht zu vergessen, dass aufgrund der mangelnden Anpassbarkeit auf eigene Datensätze eine Custom-Implementierung von \textit{SSD300} gewählt wurde und diese ebenso an vielen Stellen noch angepasst werden musste. Mit dem Inferenzverhalten als letztes Bewertungskriterium lässt sich aussagen, dass bei den behandelten Detektionssituationen des Öfteren bei beiden Detektoren die gleichen Probleme auftreten, die vermutlich, bis auf das Detektieren von verdeckten Objekten, durch eine Erweiterung des Datensatzes behoben werden könnten.

\section{Machbarkeit des Smart Warehouse Szenarios}

Nachdem sich in Kapitel \ref{realisierung} und \ref{evaluation} ausgiebig mit der Implementierung des \textit{Smart Warehouse} Szenarios und der Darstellung der Ergebnisse bezüglich der Objektdetektoren beschäftigt wurde, ergibt sich nun ein klares Bild über die Machbarkeit des \textit{Smart Warehouse} Szenarios. Es ergeben sich drei Problematiken, die nicht gelöst werden konnten: 

\begin{itemize}
	\item Das Inferieren des Video-Streams der Drohne, aufgrund der Inkompatibilität der Laufzeitumgebungen für den H.264 Decoder und der \textit{Deep Learning} Modelle,
	\item Das Detektieren von verdeckten Objekten und
	\item Das eindeutige Zählen von Objekten während der Inventur.
\end{itemize}

Um wegen erste Problematik trotzdem eine abgewandelte Variante des \textit{Smart Warehouse} Szenarios ermöglichen zu können, wurde anstelle eines Drohnen Video-Streams der Video-Stream einer Webcam verwendet. Aufgrund der oben ausgewerteten Ergebnisse wurde sich für eine Nutzung des \textit{YOLOv3} Modells für das \textit{Smart Warehouse} Szenario entschieden. Ein großer Aspekt bei der Wahl von \textit{YOLOv3} stellt hierbei die Erweiterbarkeit des Szenarios dar. Momentan ist das Szenario allein auf Getränkeflaschen exemplarisch ausgelegt. Kommt man zur eigentlichen Idee zurück, in großen Warenhäusern mit vielen verschiedenen Produkten Inventuren durchzuführen, so fällt der Aspekt des besseren Präzision der Detektion für gleiche Objekte unterschiedlicher Skalierung bei \textit{SSD300} eher in den Hintergrund. Vielmehr ist es wichtig ein zuverlässiges Detektionsverhalten zu ermöglichen. Dies ist insbesondere dann wichtig, wenn Objekte durch andere Objekte verdeckt sind und trotzdem erkannt werden sollen. Die Flugsequenz der Drohne muss für solche Szenarien hingehend optimiert werden alternative Betrachtungswinkel in Erwägung ziehen. Erweitert man zusätzlich den Datensatz auf ein solch größeres Szenario, so könnten gerade solch klassische Probleme, also Objektdetektion bei größere Entfernungen, Verdeckungsgraden und Blicklagen besser bewältigt werden. Da davon auszugehen ist, dass Warenhäuser eher gut belichtet sind, sollten extreme Beleuchtungseinflüsse ausgeschlossen werden können. Auch sind Objekte in Warenhäusern wohl kaum in Bewegung, weshalb auch die schnellere Inferenz von \textit{SSD300} bei komplexeren Datensätzen vernachlässigt werden kann. Letztendlich ist es beim Ausbau eines solchen Szenarios ebenso wichtig, schnell das Detektions-Modell auszubauen, was auch für \textit{YOLOv3} im Gegensatz zu \textit{SSD300} spricht. 

Mit der in den vorhergegangenen Kapiteln aufgezeigten Infrastruktur auf Basis einer Webcam anstelle einer Drohne und der Wahl von \textit{YOLOv3}, lässt sich nun zunächst das \textit{Smart Warehouse} Szenario einfach realisieren. Durch die Anbindung der Webcam an den \textit{Flask} Server werden kontinuierlich einzelne Bilder an den Server zur Inferenz weitergeleitet (siehe Abbildung \ref{yolo_inference}). 

\begin{figure}[H]
	\begin{center}
		\includegraphics[width=13cm]{Bilder/yolo_inference.png} 
		\caption{Bildauszug aus Video-Stream nach der Inferenz mit YOLO}
		\label{yolo_inference}
	\end{center}
\end{figure}

Problem bei der Inferenz stellt wie bereits erläutert nach wie vor die Detektion verdeckter Objekte dar. Beispielsweise sind Flaschen in Getränkekästen nur schwer detektierbar. Die Teilmerkmale verdeckter Objekte scheinen nicht ausreichend genug für eine zuversichtliche Detektion zu sein. Bei Herabsenken des \textit{confidence scores} zur Detektion gerade solcher Objekte werden wiederum zuvor doppelt detektierte Objekte erneut doppelt erkannt. Es lässt sich hierbei kein gutes Gleichgewicht einstellen.

Die letzte offene Problematik beruht nicht auf technischer Natur, sondern auf der Auswahl des Verfahrens. Der Zählalgorithmus kann das selbe Objekt bei größeren zeitlichen Intervallen oder bei unterschiedlichen absoluten Positionen auf dem Bild nicht als das selbe Objekt identifizieren. Demnach werden Objekte in solchen Szenarien doppelt gezählt. Solange allerdings Objekte nur durch das Bild \glqq gleiten\grqq{} und nicht erneut nach einem gewissen Intervall im Bild zu finden sind, lassen sich Objekte präzise zählen. Das Problem der doppelten Detektion lässt sich also nicht komplett mit dem Verfahren der Objektdetektion lösen, sondern nur auf konventionelle Verfahren mittels Scanning von Barcodes oder RFID-Chips. 

Das \textit{Smart Warehouse} Szenario ist somit nach den getroffenen Entscheidungen und mit den gegebenen Rahmenbedingungen nicht umsetzbar.



\chapter{Zusammenfassung und Ausblick}

\begin{itemize}
	\item Klare Darstellung, was die Arbeit geliefert hat
	\item Ca. 2-4 Anpunkte: Zukünftige Ziele 
\end{itemize}


% ---- Literaturverzeichnis
\cleardoublepage
\renewcommand*{\chapterpagestyle}{plain}
\pagestyle{plain}
\pagenumbering{Roman}                   % Römische Seitenzahlen
\setcounter{page}{\numexpr\value{savepage}+1}

\printbibliography[title=Literaturverzeichnis]
%\chapter*{Literaturverzeichnis}
%\addcontentsline{toc}{chapter}{Literaturverzeichnis}
%\printbibheading 
%\printbibliography[keyword=book,heading=subbibliography,title={Literaturquellen}] % Ausgabe der Literaturquellen
%\printbibliography[keyword=online,heading=subbibliography,title={Internetquellen}]     % Ausgabe der Internetquellen

% ---- Anhang

\newgeometry{
	left=2.5cm,
	right=2.5cm,
	top=0cm,
	bottom=2.5cm,
	includeheadfoot
}

\appendix
\chapter{Anhang}
%\captionsetup{list=false}

\section*{Abbildungen}



\restoregeometry
\newpage

\onehalfspacing
%\include{Inhalt/12_Anhang/...}

%\clearpage
%\pagenumbering{Roman}  % römische Seitenzahlen für Anhang

\end{document}
